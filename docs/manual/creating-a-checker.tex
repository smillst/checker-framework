\htmlhr
\chapterAndLabel{How to create a new checker}{creating-a-checker}
\label{writing-a-checker} % for old links; don't use any more!

\newcommand{\TreeAPIBase}{https://docs.oracle.com/en/java/javase/11/docs/api/jdk.compiler/com/sun/source}
\newcommand{\refTreeclass}[2]{\href{\TreeAPIBase{}/#1/#2.html?is-external=true}{\<#2>}}
\newcommand{\ModelAPIBase}{https://docs.oracle.com/en/java/javase/11/docs/api/java.compiler/javax/lang/model}
\newcommand{\refModelclass}[2]{\href{\ModelAPIBase{}/#1/#2.html?is-external=true}{\<#2>}}

This chapter describes how to create a checker
--- a type-checking compiler plugin that detects bugs or verifies their
absence.  After a programmer annotates a program,
the checker verifies that the code is consistent
with the annotations.
If you only want to \emph{use} a checker, you do not need to read this
chapter.
People who wish to edit the Checker Framework source code or
make pull requests should read the
\ahref{https://checkerframework.org/manual/developer-manual.html}{Checker
  Framework Developer Manual}.


Writing a simple checker is easy!  For example, here is a complete, useful
type-checker:

\begin{Verbatim}
import java.lang.annotation.Documented;
import java.lang.annotation.Target;
import java.lang.annotation.ElementType;
import org.checkerframework.common.subtyping.qual.Unqualified;
import org.checkerframework.framework.qual.SubtypeOf;

@Documented
@Target({ElementType.TYPE_USE, ElementType.TYPE_PARAMETER})
@SubtypeOf(Unqualified.class)
public @interface Encrypted {}
\end{Verbatim}

This checker is so short because it builds on the Subtyping Checker
(Chapter~\ref{subtyping-checker}).
See Section~\ref{subtyping-example} for more details about this particular checker.
When you wish to create a new checker, it is often easiest to begin by
building it declaratively on top of the Subtyping Checker, and then return to
this chapter when you need more expressiveness or power than the Subtyping
Checker affords.

Three choices for creating your own checker are:
\begin{itemize}
\item
  Customize an existing checker.
  Checkers that are designed for extension include
  the Subtyping Checker (\chapterpageref{subtyping-checker}),
  the Accumulation Checker (\chapterpageref{accumulation-checker}),
  the Fake Enumeration Checker (\chapterpageref{fenum-checker}),
  and the Units Checker (\chapterpageref{units-checker}).
\item
  Follow the instructions in this chapter to create a checker from scratch.
  This enables creation of checkers that are more powerful than customizing
  an existing checker.
\item
  Copy and then modify a different existing checker --- whether
  one distributed with the Checker Framework or a third-party one.
  You can get tangled up if you don't fully understand
  the subtleties of the existing checker that you are modifying.
  Usually, it is easier to follow the instructions in this chapter.
  (If you are going to copy a checker, one good choice to copy and modify
  is the Regex Checker (\chapterpageref{regex-checker}).  A bad choice is
  the Nullness Checker (\chapterpageref{nullness-checker}),
  which is more sophisticated than anything you want to start out building.)
\end{itemize}

You do not need all of the details in this chapter, at least at first.
In addition to reading this chapter of the manual, you may find it helpful
to examine the implementations of the checkers that are distributed with
the Checker Framework.
The Javadoc documentation of the framework and the checkers is in the
distribution and is also available online at
\myurl{https://checkerframework.org/api/}.

If you write a new checker and wish to advertise it to the world, let us
know so we can mention it in \chapterpageref{third-party-checkers}
or even include it in the Checker Framework distribution.


\sectionAndLabel{How checkers build on the Checker Framework}{creating-tool-relationships}

This table shows the relationship among tools that the Checker Framework
builds on or that are built on the Checker Framework.
You use the Checker Framework to build pluggable type systems, and the
Annotation File Utilities to manipulate \code{.java} and \code{.class} files.

\newlength{\bw}
\setlength{\bw}{.5in}

%% Strictly speaking, "Subtyping Checker" should sit on top of Checker
%% Framework and below all the specific checkers.  But omit it for simplicity.

% Unfortunately, Hevea inserts a horizontal line between every pair of rows
% regardless of whether there is a \hline or \cline.  So, make paragraphs.
\begin{center}
\begin{tabular}{|p{\bw}|p{\bw}|p{\bw}|p{\bw}|p{.4\bw}|p{\bw}|p{1.5\bw}|p{1\bw}|}
\cline{1-4} \cline{6-6}
\centering Subtyping \par Checker &
\centering Nullness \par Checker &
\centering Index \par Checker &
\centering Tainting \par Checker &
\centering \ldots &
\centering Your \par Checker &
\multicolumn{2}{c}{}
\\ \hline
\multicolumn{6}{|p{6\bw}|}{\centering Base Checker \par (enforces subtyping rules)} &
\centering Type \par inference &
% Adding "\centering" here causes a LaTeX alignment error
Other \par tools
\\ \hline
\multicolumn{6}{|p{6\bw}|}{\centering Checker Framework \par (enables creation of pluggable type-checkers)} &
\multicolumn{2}{p{3\bw}|}{\centering \href{https://checkerframework.org/annotation-file-utilities/}{Annotation File Utilities} \par (\code{.java} $\leftrightarrow$ \code{.class} files)}
\\ \hline
\multicolumn{8}{|p{8.5\bw}|}{\centering
  \href{https://checkerframework.org/jsr308/}{Java type annotations} syntax
  and classfile format \par \centering (no built-in semantics)} \\ \hline
\end{tabular}
\end{center}


The Base Checker
(more precisely, the \refclass{common/basetype}{BaseTypeChecker})
enforces the standard subtyping rules.
The Subtyping Checker is a simple use of the Base Checker that supports
providing type qualifiers on the command line.
You usually want to build your checker on the Base Checker.


\sectionAndLabel{The parts of a checker}{creating-parts-of-a-checker}

The Checker Framework provides abstract base classes (default
implementations), and a specific checker overrides as little or as much of
the default implementations as necessary.
To simplify checker implementations, by default the Checker Framework
automatically discovers the parts of a checker by looking for specific files.
Thus, checker implementations follow a very formulaic structure.
To illustrate, a checker for MyProp must be laid out as follows:
%
\begin{Verbatim}
myPackage/
  | qual/                               type qualifiers
  | MyPropChecker.java                  interface to the compiler
  | MyPropVisitor.java                  [optional] type rules
  | MyPropAnnotatedTypeFactory.java     [optional] type introduction and dataflow rules
\end{Verbatim}
%
\<MyPropChecker.java> is occasionally optional, such as if you are
building on the Subtyping Checker.  If you want to create an artifact
containing just the qualifiers (similar to the Checker Framework's
\<checker-qual> artifact), you should put the \<qual/> directory in a
separate Maven module or Gradle subproject.

Sections~\ref{creating-typequals}--\ref{creating-dataflow} describe
the individual components of a type system as written using the Checker
Framework:

\begin{description}

\item{\ref{creating-typequals}}
  \textbf{Type qualifiers and hierarchy.}  You define the annotations for
  the type system and the subtyping relationships among qualified types
  (for instance, \<@NonNull Object> is a subtype of \<@Nullable
  Object>).  This is also where you specify the default annotation that
  applies whenever the programmer wrote no annotation and no other defaulting
  rule applies.

\item{\ref{creating-compiler-interface}}
  \textbf{Interface to the compiler.}  The compiler interface indicates
  which annotations are part of the type system, which command-line options
  and \<@SuppressWarnings> annotations the checker recognizes, etc.

\item{\ref{creating-extending-visitor}}
  \textbf{Type rules.}  You specify the type system semantics (type
  rules), violation of which yields a type error.  A type system has two types of
  rules.
\begin{itemize}
\item
  Subtyping rules related to the type hierarchy, such as that in every
  assignment,
  % and pseudo-assignment
  the type of the right-hand-side is a subtype of the type of the left-hand-side.
  Your checker automatically inherits these subtyping rules from the Base
  Checker (Chapter~\ref{subtyping-checker}), so there is nothing for you to do.
\item
  Additional rules that are specific to your particular checker.  For
  example, in the Nullness type system, only references whose type is
  \refqualclass{checker/nullness/qual}{NonNull} may be dereferenced.  You
  write these additional rules yourself.
\end{itemize}

\item{\ref{creating-type-introduction}}
  \textbf{Type introduction rules.}  You specify the type of some expressions where
  the rules differ from the built-in framework rules.

\item{\ref{creating-dataflow}}
  \textbf{Dataflow rules.}  These optional rules enhance flow-sensitive
  type qualifier inference (also sometimes called ``local variable type inference'').
\end{description}




\sectionAndLabel{Compiling and using a custom checker}{creating-compiling}

You can place your checker's source files wherever you like.
One choice is to write your checker in a fork of the Checker Framework
repository \url{https://github.com/typetools/checker-framework}.
Another choice is to write it in a stand-alone repository.  Here is a
template for a stand-alone repository:
\url{https://github.com/typetools/templatefora-checker}; at that URL,
click the ``Use this template'' button.

% You may also wish to consult Section~\ref{creating-testing-framework} for
% information on testing a checker and
% Section~\ref{creating-debugging-options} for information on debugging a
% checker.

Once your custom checker is written, using it is very similar to using a
built-in checker (Section~\ref{running}):
simply pass the fully-qualified name of your \<BaseTypeChecker>
subclass to the \<-processor> command-line option:
\begin{alltt}
  javac -processor \textit{mypackage.MyPropChecker} SourceFile.java
\end{alltt}
Note that your custom checker's
\<.class> files must be on the same path (the classpath or processorpath)
as the Checker Framework.
Invoking a custom checker that builds on
the Subtyping Checker is slightly different (Section~\ref{subtyping-using}).



\sectionAndLabel{Tips for creating a checker}{creating-tips}

To make your job easier, we recommend that you build your type-checker
incrementally, testing at each phase rather than trying to build the whole
thing at once.

Here is a good way to proceed.

\begin{enumerate}
\item
\label{creating-tips-write-manual}
  Write the user manual.  Do this before you start coding.  The manual
  explains the type system, what it guarantees, how to use it, etc., from
  the point of view of a user.  Writing the manual will help you flesh out
  your goals and the concepts, which are easier to understand and change in
  text than in an implementation.
  Section~\ref{creating-documenting-a-checker} gives a suggested structure
  for the manual chapter, which will help you avoid omitting any parts.
  Get feedback from someone else at this point to ensure that your manual
  is comprehensible.

  Once you have designed and documented the parts of your type system, you
  should ``play computer'', manually
  type-checking some code according to the rules you defined.
  During manual checking, ask
  yourself what reasoning you applied, what information you needed, and
  whether your written-down rules were sufficient.
  It is more efficient to find problems now rather than after coding up
  your design.

\item
\label{creating-tips-implement-qualifiers}
  Implement the type qualifiers and hierarchy
  (Section~\ref{creating-typequals}).

  Write simple test cases that consist of only assignments,
  to test your type hierarchy.  For instance, if
  your type hierarchy consists of a supertype \<@UnknownSign> and a subtype
  \<@NonNegative>, then you could write a test case such as:

\begin{Verbatim}
  void testHierarchy(@UnknownSign int us, @NonNegative int nn) {
    @UnknownSign int a = us;
    @UnknownSign int b = nn;
    // :: error: assignment
    @NonNegative int c = us;  // expected error on this line
    @NonNegative int d = nn;
  }
\end{Verbatim}

  Type-check your test files using the Subtyping Checker
  (\chapterpageref{subtyping-checker}).

\item
\label{creating-tips-implement-checker}
  Write the checker class itself
  (Section~\ref{creating-compiler-interface}).

  Ensure that you can still type-check your test files and that the results
  are the same.  You will not use the Subtyping Checker any more; you will
  call the checker directly, as in

\begin{Verbatim}
  javac -processor mypackage.MyChecker File1.java File2.java ...
\end{Verbatim}

\item
\label{creating-tips-test-infrastructure}
  Test infrastructure.
  If your checker source code is in a clone of the Checker Framework
  repository, integrate your checker with the Checker Framework's Gradle
  targets for testing (Section~\ref{creating-testing-framework}).  This
  will make it much more convenient to run tests, and to ensure that they
  are passing, as your work proceeds.

\item
  Annotate parts of the JDK, if relevant
  (Section~\ref{creating-a-checker-annotated-jdk}).

  Write test cases for a few of the annotated JDK methods to ensure
  that the annotations are being properly read by your checker.

\item
  Implement type rules, if any (Section~\ref{creating-extending-visitor}).
  (Some type systems need JDK annotations but don't have any additional
  type rules.)

  Before implementing type rules (or any other code in your type-checker),
  read the Javadoc to familiarize yourself with the utility routines in the
  \<org.checkerframework.javacutil> package, especially
  \refclass{javacutil}{AnnotationBuilder},
  \refclass{javacutil}{AnnotationUtils},
  \refclass{javacutil}{ElementUtils},
  \refclass{javacutil}{TreeUtils},
  \refclass{javacutil}{TypeAnnotationUtils}, and
  \refclass{javacutil}{TypesUtils}.
  You will learn how to access needed information and avoid
  reimplementing existing functionality.

  Write simple test cases to test the type rules, and ensure that the
  type-checker behaves as expected on those test files.
  For example, if your type system forbids indexing an array by a
  possibly-negative value, then you would write a test case such as:

\begin{Verbatim}
  void testArrayIndexing(String[] myArray, @UnknownSign int us, @NonNegative int nn) {
    myArray[us];  // expected error on this line
    myArray[nn];
  }
\end{Verbatim}

\item
  Implement type introduction rules, if any (Section~\ref{creating-type-introduction}).

  Test your type introduction rules.
  For example, if your type system sets the qualifier for manifest literal
  integers and for array lengths, you would write a test case like the following:

\begin{Verbatim}
  void testTypeIntroduction(String[] myArray) {
    @NonNegative nn1 = -1;  // expected error on this line
    @NonNegative nn2 = 0;
    @NonNegative nn3 = 1;
    @NonNegative nn4 = myArray.length;
  }
\end{Verbatim}

\item
  Optionally, implement dataflow refinement rules
  (Section~\ref{creating-dataflow}).

  Test them if you wrote any.
  For instance, if after an arithmetic comparison, your type system infers
  which expressions are now known to be non-negative, you could write a
  test case such as:

\begin{Verbatim}
  void testDataflow(@UnknownSign int us, @NonNegative int nn) {
    @NonNegative nn2;
    nn2 = us;  // expected error on this line
    if (us > j) {
      nn2 = us;
    }
    if (us >= j) {
      nn2 = us;
    }
    if (j < us) {
      nn2 = us;
    }
    if (j <= us) {
      nn2 = us;
    }
    nn = us;  // expected error on this line
  }
\end{Verbatim}

\end{enumerate}




\sectionAndLabel{Annotations: Type qualifiers and hierarchy}{creating-typequals}

A type system designer specifies the qualifiers in the type system (Section~\ref{creating-define-type-qualifiers})
and
the type hierarchy that relates them.
The type hierarchy --- the subtyping relationships among the qualifiers ---
can be defined either
declaratively via meta-annotations (Section~\ref{creating-declarative-hierarchy}), or procedurally through
subclassing \refclass{framework/type}{QualifierHierarchy} or
\refclass{framework/type}{TypeHierarchy} (Section~\ref{creating-procedural-hierarchy}).


\subsectionAndLabel{Defining the type qualifiers}{creating-define-type-qualifiers}

%% True, but seems irrelevant here, so it detracts from the message.
% Each qualifier restricts the values that
% a type can represent.  For example \<@NonNull String> type can only
% represent non-null values, indicating that the variable may not hold
% \<null> values.

Type qualifiers are defined as Java annotations.  In Java, an
annotation is defined using the Java \code{@interface} keyword.
Here is how to define a two-qualifier hierarchy:

\begin{Verbatim}
package mypackage.qual;
import java.lang.annotation.Documented;
import java.lang.annotation.ElementType;
import java.lang.annotation.Retention;
import java.lang.annotation.RetentionPolicy;
import java.lang.annotation.Target;
import org.checkerframework.framework.qual.DefaultQualifierInHierarchy;
import org.checkerframework.framework.qual.SubtypeOf;
/**
 * The run-time value of the integer is unknown.
 *
 * @checker_framework.manual #nonnegative-checker Non-Negative Checker
 */
@Documented
@Retention(RetentionPolicy.RUNTIME)
@Target({ElementType.TYPE_USE, ElementType.TYPE_PARAMETER})
@SubtypeOf({})
@DefaultQualifierInHierarchy
public @interface UnknownSign {}


package mypackage.qual;
import java.lang.annotation.Documented;
import java.lang.annotation.ElementType;
import java.lang.annotation.Retention;
import java.lang.annotation.RetentionPolicy;
import java.lang.annotation.Target;
import org.checkerframework.framework.qual.LiteralKind;
import org.checkerframework.framework.qual.SubtypeOf;
/**
 * Indicates that the value is greater than or equal to zero.
 *
 * @checker_framework.manual #nonnegative-checker Non-Negative Checker
 */
@Documented
@Retention(RetentionPolicy.RUNTIME)
@Target({ElementType.TYPE_USE, ElementType.TYPE_PARAMETER})
@SubtypeOf({UnknownSign.class})
public @interface NonNegative {}
\end{Verbatim}

The \refqualclass{framework/qual}{SubtypeOf} meta-annotation
indicates the parent in the type hierarchy.

The \sunjavadocanno{java.base/java/lang/annotation/Target.html}{Target}
meta-annotation indicates where the annotation
may be written. All type qualifiers that users can write in source code should
have the value \<ElementType.TYPE\_USE> and optionally with the additional value
of \<ElementType.TYPE\_PARAMETER>, but no other \<ElementType> values.
%% This feels like clutter that distracts from the main point of the section.
% (Terminological note:  a \emph{meta-annotation} is an annotation that
% is written on an annotation definition, such as
% \refqualclass{framework/qual}{SubtypeOf} and
% \sunjavadocanno{java.base/java/lang/annotation/Target.html}{Target}.)

The annotations should be placed within a directory called \<qual>, and
\<qual> should be placed in the same directory as your checker's source file.
The Checker Framework automatically treats any annotation that
is declared in the \<qual> package as a type qualifier.
(See Section \ref{creating-indicating-supported-annotations} for more details.)
For example, the Nullness Checker's source file is located at
\<.../nullness/NullnessChecker.java>. The \<@NonNull> qualifier is defined in
file \<.../nullness/qual/NonNull.java>.

% \noindent
% The \<@Target({ElementType.TYPE\_USE})> meta-annotation
% distinguishes it from an ordinary
% annotation that applies to a declaration (e.g., \<@Deprecated> or
% \<@Override>).
% The framework ignores any annotation whose
% declaration does not bear the \<@Target({ElementType.TYPE\_USE})>
% meta-annotation (with minor
% exceptions, such as \<@SuppressWarnings>).

Your type system should include a top qualifier and a bottom qualifier
(Section~\ref{creating-bottom-and-top-qualifier}).
The top qualifier is conventionally named \<\emph{CheckerName}Unknown>.
Most type systems should also include a
polymorphic qualifier \<@Poly\emph{MyTypeSystem}>
(Section~\ref{method-qualifier-polymorphism}).

Choose good names for the qualifiers, because users will write these in
their source code.
The Javadoc of every type qualifier should include a precise English
description and an example use of the qualifier.


\subsectionAndLabel{Declaratively defining the qualifier hierarchy}{creating-declarative-hierarchy}

Declaratively, the type system designer uses two meta-annotations (written
on the declaration of qualifier annotations) to specify the qualifier
hierarchy.

\begin{itemize}

\item \refqualclass{framework/qual}{SubtypeOf} denotes that a qualifier is a subtype of
  another qualifier or qualifiers, specified as an array of class
  literals.  For example, for any type $T$,
  \refqualclass{checker/nullness/qual}{NonNull} $T$ is a subtype of \refqualclass{checker/nullness/qual}{Nullable} $T$:

  \begin{Verbatim}
    @Target({ElementType.TYPE_USE, ElementType.TYPE_PARAMETER})
    @SubtypeOf( { Nullable.class } )
    public @interface NonNull {}
  \end{Verbatim}

  % (The actual definition of \refclass{checker/nullness/qual}{NonNull} is slightly more complex.)


  %% True, but a distraction.  Move to Javadoc?
  % (It would be more natural to use Java subtyping among the qualifier
  % annotations, but Java forbids annotations from subtyping one another.)
  %
  \refqualclass{framework/qual}{SubtypeOf} accepts multiple annotation classes as an argument,
  permitting the type hierarchy to be an arbitrary DAG\@.

% TODO: describe multiple type hierarchies
% TODO: describe multiple polymorphic qualifiers
% TODO: the code consistently uses "top" for type qualifiers and
%       "root" for ASTs, in particular for CompilationUnitTrees.

  All type qualifiers, except for polymorphic qualifiers (see below and
  also Section~\ref{method-qualifier-polymorphism}), need to be
  properly annotated with \refclass{framework/qual}{SubtypeOf}.

  The top qualifier is annotated with
  \<@SubtypeOf( \{ \} )>.  The top qualifier is the qualifier that is
  a supertype of all other qualifiers.  For example, \refqualclass{checker/nullness/qual}{Nullable}
  is the top qualifier of the Nullness type system, hence is defined as:

  \begin{Verbatim}
    @Target({ElementType.TYPE_USE, ElementType.TYPE_PARAMETER})
    @SubtypeOf( {} )
    public @interface Nullable {}
  \end{Verbatim}

  \begin{sloppypar}
  If the top qualifier of the hierarchy is the generic unqualified type
  (this is not recommended!), then its children
  will use \code{@SubtypeOf(Unqualified.class)}, but no
  \code{@SubtypeOf(\{\})} annotation on the top qualifier \<Unqualified> is
  necessary.  For an example, see the
  \<Encrypted> type system of Section~\ref{encrypted-example}.
  \end{sloppypar}

\item \refqualclass{framework/qual}{PolymorphicQualifier} denotes that a qualifier is a
  polymorphic qualifier.  For example:

  \begin{Verbatim}
    @Target({ElementType.TYPE_USE, ElementType.TYPE_PARAMETER})
    @PolymorphicQualifier
    public @interface PolyNull {}
  \end{Verbatim}

  For a description of polymorphic qualifiers, see
  Section~\ref{method-qualifier-polymorphism}.  A polymorphic qualifier must not have
  a \refqualclass{framework/qual}{SubtypeOf} meta-annotation nor be
  mentioned in any other \refqualclass{framework/qual}{SubtypeOf}
  meta-annotation.

\end{itemize}

The declarative and procedural mechanisms for specifying the hierarchy can
be used together. If any of the annotations representing type qualifiers have elements, then
the relationships between those qualifiers must be defined procedurally.


\subsectionAndLabel{Procedurally defining the qualifier hierarchy}{creating-procedural-hierarchy}

The declarative syntax suffices for most cases.  More complex type
hierarchies can be expressed by overriding, in your subclass of
\refclass{common/basetype}{BaseAnnotatedTypeFactory}, either
\refmethodterse{framework/type}{AnnotatedTypeFactory}{createQualifierHierarchy}{--}
or \refmethodterse{framework/type}{AnnotatedTypeFactory}{createTypeHierarchy}{--}
(typically only one of these needs to be overridden).

For more details, see the Javadoc of those methods and of the classes
\refclass{framework/type}{QualifierHierarchy} and \refclass{framework/type}{TypeHierarchy}.

The \refclass{framework/type}{QualifierHierarchy} class represents the qualifier hierarchy (not the
type hierarchy).  A type-system designer may subclass
\refclass{framework/type}{QualifierHierarchy} to express customized qualifier
relationships (e.g., relationships based on annotation
arguments).

The \refclass{framework/type}{TypeHierarchy} class represents the type hierarchy ---
that is, relationships between
annotated types, rather than merely type qualifiers, e.g., \<@NonNull
Date> is a subtype of \<@Nullable Date>.  The default \refclass{framework/type}{TypeHierarchy} uses
\refclass{framework/type}{QualifierHierarchy} to determine all subtyping relationships.
The default \refclass{framework/type}{TypeHierarchy} handles
generic type arguments, array components, type variables, and
wildcards in a similar manner to the Java standard subtype
relationship but with taking qualifiers into consideration.  Some type
systems may need to override that behavior.  For instance, the Java
Language Specification specifies that two generic types are subtypes only
if their type arguments are identical:  for example,
\code{List<Date>} is not a subtype of \code{List<Object>}, or of any other
generic \code{List}.
(In the technical jargon, the generic arguments are ``invariant'' or ``novariant''.)


\subsectionAndLabel{Defining the default annotation}{creating-typesystem-defaults}

A type system applies the default qualifier where the user has not written a
qualifier (and no other default qualifier is applicable), as explained in
Section~\ref{defaults}.

The type system designer must specify the default annotation. The designer can specify the default annotation declaratively,
using the \refqualclass{framework/qual}{DefaultQualifierInHierarchy}
meta-annotation.
Note that the default will apply to any source code that the checker reads,
including stub libraries, but will not apply to compiled \code{.class}
files that the checker reads.

\begin{sloppypar}
Alternately, the type system designer may specify a default procedurally,
by overriding the
\refmethod{framework/type}{GenericAnnotatedTypeFactory}{addCheckedCodeDefaults}{-org.checkerframework.framework.util.defaults.QualifierDefaults-}
method.  You may do this even if you have declaratively defined the
qualifier hierarchy.
\end{sloppypar}

If the default qualifier in the type hierarchy requires a value, there are
ways for the type system designer to specify a default value both
declaratively and procedurally, as well.  To do so declaratively, append
the string \<default \emph{value}> where \emph{value} is the actual value
you want to be the default, after the declaration of the value in the
qualifier file.  For instance, \code{int value() default 0;} would make
\code{value} default to zero. Alternatively, the procedural method
described above can be used.

The default qualifier applies to most, but not all, unannotated types, Section~\ref{climb-to-top}
other defaulting rules are automatically added to every checker. Also, Section~\ref{defaults}
describes other meta-annotations used to specify default annotations.

\subsectionAndLabel{Relevant Java types}{creating-relevant-java-types}

Sometimes, a checker only processes certain Java types.  For example, the
\ahrefloc{formatter-checker}{Format String Checker} is relevant only to
\<CharSequence> and its subtypes such as \<String>.
The \refqualclass{framework/qual}{RelevantJavaTypes}
annotation on the checker class indicates that its qualifiers may only be
written on those types and no others.  All irrelevant types are defaulted to
the top annotation.


\subsectionAndLabel{Do not re-use type qualifiers}{creating-do-not-re-use-type-qualifiers}

Every annotation should belong to only one type system.  No annotation
should be used by multiple type systems.  This is true even of annotations
that are internal to the type system and are not intended to be written by
the programmer.

Suppose that you have two type systems that both use the same type
qualifier \<@Bottom>.  In a client program, a use of type \<T> may require type
qualifier \<@Bottom> for one type system but a different qualifier for the other
type system.  There is no annotation that a programmer can write to make
the program type-check under both type systems.

This also applies to type qualifiers that a programmer does not write,
because the compiler outputs \<.class> files that contain an explicit type
qualifier on every type --- a defaulted or inferred type qualifier if the
programmer didn't write a type qualifier explicitly.


\subsectionAndLabel{Completeness of the type hierarchy}{creating-bottom-and-top-qualifier}

When you define a type system, its type hierarchy must be a
lattice:  every set of types has a unique least upper bound and a unique
greatest lower bound.  This implies that there must be a top type that is a
supertype of all other types, and there must be a bottom type that is a
subtype of all other types.
Furthermore, the top type and bottom type should be defined
specifically for the type system.  Don't reuse an existing qualifier from the
Checker Framework such as \<@Unqualified>.

It is possible that a single type-checker checks multiple type hierarchies.
An example is the Nullness Checker, which has three separate type
hierarchies, one each for
nullness, initialization, and map keys.  In this case, each type hierarchy
would have its own top qualifier and its own bottom qualifier; they don't
all have to share a single top qualifier or a single bottom qualifier.


\paragraphAndLabel{Bottom qualifier}{creating-bottom-qualifier}
Your type hierarchy must have a bottom qualifier
--- a qualifier that is a (direct or indirect) subtype of every other
qualifier.

\<null> is the bottom type. Because the only value with type \<Void> is
\<null>, uses of the type \<Void> are also bottom.
(The only exception
is if the type system has special treatment for \<null> values, as the
Nullness Checker does. In that case, add the meta-annotation
\refqualclasswithparams{framework/qual}{QualifierForLiterals}{LiteralKind.NULL}
to the correct qualifier.)
This legal code
will not type-check unless \<null> has the bottom type:
\begin{Verbatim}
<T> T f() {
    return null;
}
\end{Verbatim}

% \begin{sloppypar}
% You don't necessarily have to define a new bottom qualifier.  You can
% use \<org.checkerframework.common.subtyping.qual.Bottom> if your type system does not already have an
% appropriate bottom qualifier.
% \end{sloppypar}

Some type systems have a special bottom type that is used \emph{only} for
the \code{null} value, and for dead code and other erroneous situations.
In this case, users should only write the bottom qualifier on explicit
bounds.  In this case, the definition of the bottom qualifier should be
meta-annotated with:

% import java.lang.annotation.ElementType;
% import java.lang.annotation.Target;
% import org.checkerframework.framework.qual.TargetLocations;
% import org.checkerframework.framework.qual.TypeUseLocation;
%
\begin{Verbatim}
@Target({ElementType.TYPE_USE, ElementType.TYPE_PARAMETER})
@TargetLocations({TypeUseLocation.EXPLICIT_LOWER_BOUND, TypeUseLocation.EXPLICIT_UPPER_BOUND})
\end{Verbatim}

Furthermore, by convention the name of such a qualifier ends with ``\<Bottom>''.

The hierarchy shown in Figure~\ref{fig-initialization-hierarchy} lacks
a bottom qualifier, but the actual implementation does contain a (non-user-visible) bottom qualifier.


\paragraphAndLabel{Top qualifier}{creating-top-qualifier}
Your type hierarchy must have a top qualifier
--- a qualifier that is a (direct or indirect) supertype of every other
qualifier.
Here is one reason.
The default type for local variables is the top
qualifier (that type is then flow-sensitively
refined depending on what values are stored in the local variable).
If there is no single top qualifier, then there is no
unambiguous choice to make for local variables.


\subsectionAndLabel{Annotations whose argument is a Java expression (dependent type annotations)\label{expression-annotations}}{dependent-types}

Sometimes, an annotation needs to refer to a Java expression.
Section~\ref{java-expressions-as-arguments} gives examples of such
annotations and also explains what Java expressions can and cannot be
referred to.

This section explains how to implement a dependent type annotation.

A ``dependent type annotation''
must have one attribute, \<value>, that is an
array of strings.  The Checker Framework verifies that the annotation's
arguments are valid expressions according to the rules of
Section~\ref{java-expressions-as-arguments}.  If
the expression is not valid, an error is issued and the string in the
annotation is changed to indicate that the expression is not valid.

The Checker Framework standardizes the expression strings.  For example, a
field \<f> can be referred to as either ``\<f>'' or ``\<this.f>''.  If the
programmer writes ``\<f>'', the Checker Framework treats it
as if the programmer had written ``\<this.f>''.
An advantage of this canonicalization is
that comparisons, such as \<isSubtype>, can be implemented as string comparisons.

The Checker Framework viewpoint-adapts type annotations on method, constructor,
and field declarations at uses for those methods.  For example, given the
following class

\begin{Verbatim}
class MyClass {
   Object field = ...;
   @Anno("this.field") Object field2 = ...;
}
\end{Verbatim}
and assuming the variable \<myClass> is of type \<MyClass>, then the type of
\<myClass.field> is viewpoint-adapted to \<@Anno("myClass.field")>.

To use this built-in functionality, add a \refqualclass{framework/qual}{JavaExpression} annotation
to any annotation element that should be interpreted as a Java expression.  The type of the
element must be an array of Strings.  If your checker requires special handling of Java expressions,
your checker implementation should override
\refmethod{framework/type}{GenericAnnotatedTypeFactory}{createDependentTypesHelper}{--}
to return a subclass of \<DependentTypesHelper>.

Given a specific expression in the program (of type Tree or Node), a
checker may need to obtain its canonical string representation.  This
enables the checker to create an dependent type annotation that refers to
it, or to compare to the string expression of an existing expression
annotation.
To obtain the string, first create a
\refclass{dataflow/expression}{JavaExpression} object by calling
\refmethodanchortext{dataflow/expression}{JavaExpression}{fromTree}{-com.sun.source.tree.ExpressionTree-}{fromTree(AnnotationProvider,
ExpressionTree)} or
\refmethodanchortext{dataflow/expression}{JavaExpression}{fromNode}{-org.checkerframework.dataflow.cfg.node.Node-}{fromNode(AnnotationProvider,
Node)}.
Then, call \<toString()> on the \<JavaExpression> object.


\subsectionAndLabel{Repeatable annotations}{repeatable-annotations}

Some pre- and post-condition annotations that have multiple elements (that
is, annotations that take multiple arguments) should be repeatable, so that
programmers can specify them more than once.  An example is
\refqualclass{checker/nullness/qual}{EnsuresNonNullIf}; it could not be
defined with each of its elements being a list, as (for example)
\refqualclass{checker/nullness/qual}{KeyFor} is.

Make an annotation \emph{A} repeatable by defining a nested annotation (within
\emph{A}'s definition) named \<List>, and writing
\<@Repeatable(\emph{A}.List.class)> on \emph{A}'s definition.
% This convention is encoded in
% AnnotatedTypeFactory.isListForRepeatedAnnotationImplementation .


\sectionAndLabel{The checker class:  Compiler interface}{creating-compiler-interface}

A checker's entry point is a subclass of
\refclass{framework/source}{SourceChecker}, and is usually a direct subclass
of either \refclass{common/basetype}{BaseTypeChecker} or
\refclass{framework/source}{AggregateChecker}.
This entry
point, which we call the checker class, serves two
roles:  an interface to the compiler and a factory for constructing
type-system classes.

Because the Checker Framework provides reasonable defaults, oftentimes the
checker class has no work to do.  Here are the complete definitions of the
checker classes for the Interning Checker and the Nullness Checker:

\begin{Verbatim}
  package my.package;
  import org.checkerframework.common.basetype.BaseTypeChecker;
  @SupportedLintOptions({"dotequals"})
  public final class InterningChecker extends BaseTypeChecker {}

  package my.package;
  import org.checkerframework.common.basetype.BaseTypeChecker;
  @SupportedLintOptions({"flow", "cast", "cast:redundant"})
  public class NullnessChecker extends BaseTypeChecker {}
\end{Verbatim}

(The \refqualclass{framework/source}{SupportedLintOptions} annotation is
optional, and many checker classes do not have one.)

The checker class bridges between the Java compiler and the checker.  It
invokes the type-rule check visitor on every Java source file being
compiled.  The checker uses
\refmethod{framework/source}{SourceChecker}{reportError}{-java.lang.Object-java.lang.String-java.lang.Object...-}
and
\refmethod{framework/source}{SourceChecker}{reportWarning}{-java.lang.Object-java.lang.String-java.lang.Object...-}
to issue errors.

Also, the checker class follows the factory method pattern to
construct the concrete classes (e.g., visitor, factory) and annotation
hierarchy representation.  It is a convention that, for
a type system named Foo, the compiler
interface (checker), the visitor, and the annotated type factory are
named as \<FooChecker>, \<FooVisitor>, and \<FooAnnotatedTypeFactory>.
\refclass{common/basetype}{BaseTypeChecker} uses the convention to
reflectively construct the components.  Otherwise, the checker writer
must specify the component classes for construction.

\begin{sloppypar}
A checker can customize the default error messages through a
\sunjavadoc{java.base/java/util/Properties.html}{Properties}-loadable text file named
\<messages.properties> that appears in the same directory as the checker class.
The property file keys are the strings passed to \refmethodterse{framework/source}{SourceChecker}{reportError}{-java.lang.Object-java.lang.String-java.lang.Object...-}
and
\refmethodterse{framework/source}{SourceChecker}{reportWarning}{-java.lang.Object-java.lang.String-java.lang.Object...-}
(like \code{"type.incompatible"}) and the values are the strings to be
printed (\code{"cannot assign ..."}).
The \<messages.properties> file only need to mention the new messages that
the checker defines.
It is also allowed to override messages defined in superclasses, but this
is rarely needed.
Section~\refwithpageparen{compiler-message-keys} discusses best practices
when using a message key in a \<@SuppressWarnings> annotation.
\end{sloppypar}

\subsectionAndLabel{Indicating supported annotations}{creating-indicating-supported-annotations}

A checker must indicate the annotations that it supports (that make up its type
hierarchy).

By default, a checker supports all type annotations located in a
subdirectory called \<qual> that's located in the same directory as the checker.
A type annotation is meta-annotated with either
\<@Target(ElementType.TYPE\_USE)>
or
\<@Target({ElementType.TYPE\_USE, ElementType.TYPE\_PARAMETER})>.

To indicate support for annotations that are located outside of the \<qual>
subdirectory, annotations that have other \<ElementType> values,
checker writers can override the
\refmethodterse{framework/type}{AnnotatedTypeFactory}{createSupportedTypeQualifiers}{--}
method (see its Javadoc for details).
It is required to define \<createSupportedTypeQualifiers> if you are mixing
qualifiers from multiple directories (including when extending an existing
checker that has its own qualifiers) and when using the Buck build tool,
whose classloader cannot find the qualifier directory.

An aggregate checker (which extends
\refclass{framework/source}{AggregateChecker}) does not need to specify its
type qualifiers, but each of its component checkers should do so.


\subsectionAndLabel{Bundling multiple checkers}{creating-bundling-multiple-checkers}

Sometimes, multiple checkers work together and should always be run
together.  There are two different ways to bundle multiple checkers
together, by creating either an ``aggregate checker'' or a ``compound checker''.


\begin{enumerate}
\item
An aggregate checker runs multiple independent, unrelated checkers.  There
is no communication or cooperation among them.

The effect is the same as if a user passes
multiple processors to the \<-processor> command-line option.

For example, instead of a user having to run

\begin{Verbatim}
  javac -processor DistanceUnitChecker,VelocityUnitChecker,MassUnitChecker MyFile.java
\end{Verbatim}

\noindent
the user can write

\begin{Verbatim}
  javac -processor MyUnitCheckers MyFile.java
\end{Verbatim}

\noindent
if you define an aggregate checker class.  Extend \refclass{framework/source}{AggregateChecker} and override
the \<getSupportedTypeCheckers> method, like the following:

\begin{Verbatim}
  public class MyUnitCheckers extends AggregateChecker {
    protected Collection<Class<? extends SourceChecker>> getSupportedCheckers() {
      return Arrays.asList(DistanceUnitChecker.class,
                           VelocityUnitChecker.class,
                           MassUnitChecker.class);
    }
  }
\end{Verbatim}

% This is the *only* example, as of July 2015.
An example of an aggregate checker is \refclass{checker/i18n}{I18nChecker}
(see \chapterpageref{i18n-checker}), which consists of
\refclass{checker/i18n}{I18nSubchecker} and
\refclass{checker/i18n}{LocalizableKeyChecker}.

\item
Use a compound checker to express dependencies among checkers.  Suppose it
only makes sense to run MyChecker if MyHelperChecker has already been run;
that might be the case if MyHelperChecker computes some information that
MyChecker needs to use.

Override
\<MyChecker.\refmethodterse{common/basetype}{BaseTypeChecker}{getImmediateSubcheckerClasses}{--}>
to return a list of the checkers that MyChecker depends on.  Every one of
them will be run before MyChecker is run.  One of MyChecker's subcheckers
may itself be a compound checker, and multiple checkers may declare a
dependence on the same subchecker.  The Checker Framework will run each
checker once, and in an order consistent with all the dependences.

A checker obtains information from its subcheckers (those that ran before
it) by querying their \refclass{framework/type}{AnnotatedTypeFactory} to
determine the types of variables.  Obtain the \<AnnotatedTypeFactory> by
calling
\refmethodterse{common/basetype}{BaseTypeChecker}{getTypeFactoryOfSubchecker}{-java.lang.Class-}.

\end{enumerate}



\subsectionAndLabel{Providing command-line options}{creating-providing-command-line-options}

A checker can provide two kinds of command-line options:
boolean flags and
named string values (the standard annotation processor
options).

\subsubsectionAndLabel{Boolean flags}{creating-providing-command-line-options-boolean-flags}

To specify a simple boolean flag, add:

\begin{alltt}
  \refqualclass{framework/source}{SupportedLintOptions}(\{"myflag"\})
\end{alltt}

\noindent
to your checker subclass.
The value of the flag can be queried using

\begin{Verbatim}
  checker.getLintOption("myflag", false)
\end{Verbatim}

The second argument sets the default value that should be returned.

To pass a flag on the command line, call javac as follows:

\begin{Verbatim}
  javac -processor mypackage.MyChecker -Alint=myflag
\end{Verbatim}


\subsubsectionAndLabel{Named string values}{creating-providing-command-line-options-named-string-values}

For more complicated options, one can use the standard
\code{@SupportedOptions} annotation on the checker, as in:

\begin{alltt}
  \refqualclass{framework/source}{SupportedOptions}(\{"myoption"\})
\end{alltt}

The value of the option can be queried using

\begin{Verbatim}
  checker.getOption("myoption")
\end{Verbatim}

To pass an option on the command line, call javac as follows:

\begin{Verbatim}
  javac -processor mypackage.MyChecker -Amyoption=p1,p2
\end{Verbatim}

The value is returned as a single string and you have to perform the
required parsing of the option.


% TODO: describe -ANullnessChecker_option=value mechanism.


\sectionAndLabel{Visitor: Type rules}{creating-extending-visitor}

A type system's rules define which operations on values of a
particular type are forbidden.
These rules must be defined procedurally, not declaratively.
Put them in a file \<\emph{MyChecker}Visitor.java> that extends
\refclass{common/basetype}{BaseTypeVisitor}.

BaseTypeVisitor performs type-checking at each node of a
source file's AST\@.  It uses the visitor design pattern to traverse
Java syntax trees as provided by Oracle's
\href{https://docs.oracle.com/en/java/javase/11/docs/api/jdk.compiler/module-summary.html}{jdk.compiler
API},
and it issues a warning (by calling
\refmethodterse{framework/source}{SourceChecker}{reportError}{-java.lang.Object-java.lang.String-java.lang.Object...-}
or
\refmethodterse{framework/source}{SourceChecker}{reportWarning}{-java.lang.Object-java.lang.String-java.lang.Object...-})
whenever the type system is violated.

Most type-checkers
override only a few methods in \refclass{common/basetype}{BaseTypeVisitor}.
A checker's visitor overrides one method in the base visitor for each special
rule in the type qualifier system.
The last line of the overridden version is
``\<return super.visit\emph{TreeType}(node, p);>''.
If the method didn't raise any error,
the superclass implementation can perform standard checks.


By default, \refclass{common/basetype}{BaseTypeVisitor} performs subtyping checks that are
similar to Java subtype rules, but taking the type qualifiers into account.
\refclass{common/basetype}{BaseTypeVisitor} issues these errors:

\begin{itemize}

\item invalid assignment (type.incompatible) for an assignment from
  an expression type to an incompatible type.  The assignment may be a
  simple assignment, or pseudo-assignment like return expressions or
  argument passing in a method invocation

  In particular, in every assignment and pseudo-assignment, the
  left-hand side of the assignment is a supertype of (or the same type
  as) the right-hand side.  For example, this assignment is not
  permitted:

  \begin{Verbatim}
    @Nullable Object myObject;
    @NonNull Object myNonNullObject;
    ...
    myNonNullObject = myObject;  // invalid assignment
  \end{Verbatim}

\item invalid generic argument (type.argument) when a type
  is bound to an incompatible generic type variable

\item invalid method invocation (method.invocation) when a
  method is invoked on an object whose type is incompatible with the
  method receiver type

\item invalid overriding parameter type (override.param)
  when a parameter in a method declaration is incompatible with that
  parameter in the overridden method's declaration

\item invalid overriding return type (override.return) when the
  return type in a method declaration is incompatible with the
  return type in the overridden method's declaration

\item invalid overriding receiver type (override.receiver)
  when a receiver in a method declaration is incompatible with that
  receiver in the overridden method's declaration

\end{itemize}


\subsectionAndLabel{AST traversal}{creating-ast-traversal}

The Checker Framework needs to do its own traversal of the AST even though
it operates as an ordinary annotation processor~\cite{JSR269}.  Java
provides a visitor for Java code that is intended to be used by annotation
processors, but that visitor only
visits the public elements of Java code, such as classes, fields, methods,
and method arguments --- it does not visit code bodies or various other
locations.  The Checker Framework hardly uses the built-in visitor --- as
soon as the built-in visitor starts to visit a class, then the Checker
Framework's visitor takes over and visits all of the class's source code.

Because there is no standard API for the AST of Java
code\footnote{Actually, there is a standard API for Java ASTs --- JSR 198
  (Extension API for Integrated Development Environments)~\cite{JSR198}.
  If tools were to implement it (which would just require writing wrappers
  or adapters), then the Checker Framework and similar tools could be
  portable among different compilers and IDEs.}, the Checker Framework uses
the javac implementation.  This is why the Checker Framework is not deeply
integrated with Eclipse or IntelliJ IDEA, but runs as an external tool (see
Section~\ref{eclipse}).


\subsectionAndLabel{Avoid hardcoding}{creating-avoid-hardcoding}

If a method's contract is expressible in the type system's annotation
syntax, then you should write annotations, in a stub file or annotated JDK
(Chapter~\ref{annotating-libraries}).

Only if the contract is not expressible should you write a type-checking
rule for method invocation, where your rule checks the name of the method
being called and then treats the method in a special way.


\sectionAndLabel{Type factory: Type introduction rules}{creating-type-introduction}

The annotated type of expressions and types are defined via type introduction rules in the
type factory.  For most expressions and types, these rules are the same regardless of the type system.
For example, the type of a method invocation expression is the return type of the invoked method,
viewpoint-adapted for the call site.  The framework implements these rules so that all type systems
automatically use them.  For other expressions, such as string literals, their (annotated) types depend
on the type system, so the framework provides way to specify what qualifiers should apply to these expressions.

Defaulting rules are type introduction rules for computing the annotated type for an unannotated type;
these rules are explained in Section~\ref{creating-typesystem-defaults}. The meta-annotation \refqualclass{framework/qual}{QualifierForLiterals} can be written on an annotation
declaration to specify that the annotation should be applied to the type of literals listed in the
meta-annotation.

\subsectionAndLabel{Procedurally specifying type introduction rules}{creating-procedurally-specifying-implicit-annotations}

If the meta-annotations are not sufficiently expressive, then you
can write your own type introduction rules.  There are three ways to do so.
Each makes changes to an \refclass{framework/type}{AnnotatedTypeMirror},
which is the Checker Framework's representation of an annotated type.


\begin{enumerate}
\item
  Define a subclass of
  \refclass{framework/type/treeannotator}{TreeAnnotator},
  typically as a private inner class of your \<AnnotatedTypeFactory>.
  There is a method of \<TreeAnnotator> for every AST node, and the visitor
  has access to both the tree (the AST node) and its type.  In your
  subclass of \<AnnotatedTypeFactory>, override \<createTreeAnnotator> to
  return a \<ListTreeAnnotator> containing that annotator, as in

\begin{Verbatim}
  @Override
  protected TreeAnnotator createTreeAnnotator() {
      return new ListTreeAnnotator(super.createTreeAnnotator(), new MyTreeAnnotator(this));
  }
\end{Verbatim}

  \noindent
  (or put your TreeAnnotator first; several tree annotators are run by
  default, and among them, \<PropagationTreeAnnotator>
  adds annotations to \<AnnotatedTypeMirror>s that do not have an annotation,
  but has no effect on those that have an annotation).

\item
  Define a subclass of a
  \refclass{framework/type/typeannotator}{TypeAnnotator},
  typically as a private inner class of your \<AnnotatedTypeFactory>.
  There is a method of \<TypeAnnotator> for every kind of type, and the
  visitor has access to only the type.  In your subclass of
  \<AnnotatedTypeFactory>, override \<createTypeAnnotator> to return a
  \<ListTypeAnnotator> containing that annotator, as in

\begin{Verbatim}
  @Override
  protected TypeAnnotator createTypeAnnotator() {
      return new ListTypeAnnotator(new MyTypeAnnotator(this), super.createTypeAnnotator());
  }
\end{Verbatim}

  \noindent
  (or put your TypeAnnotator last).

\item
  Create a subclass of \refclass{framework/type}{AnnotatedTypeFactory} and
  override two \<addComputedTypeAnnotations> methods:
  \refmethodanchortext{framework/type}{AnnotatedTypeFactory}{addComputedTypeAnnotations}{-com.sun.source.tree.Tree-org.checkerframework.framework.type.AnnotatedTypeMirror-}{addComputedTypeAnnotations(Tree,AnnotatedTypeMirror)}
  (or
  \refmethodanchortext{framework/type}{GenericAnnotatedTypeFactory}{addComputedTypeAnnotations}{-com.sun.source.tree.Tree-org.checkerframework.framework.type.AnnotatedTypeMirror-boolean-}{addComputedTypeAnnotations(Tree,AnnotatedTypeMirror,boolean)}
  if extending \code{GenericAnnotatedTypeFactory})
  and
  \refmethodanchortext{framework/type}{AnnotatedTypeFactory}{addComputedTypeAnnotations}{-javax.lang.model.element.Element-org.checkerframework.framework.type.AnnotatedTypeMirror-}{addComputedTypeAnnotations(Element,AnnotatedTypeMirror)}.
  The methods can make arbitrary changes to the annotations on a type.

  Recall that \<AnnotatedTypeFactory>, when given a program
  expression, returns the expression's type.  This should include not only
  the qualifiers that the programmer explicitly wrote in the source code, but
  also default annotations and type
  refinement (see Section~\ref{effective-qualifier} for explanations of these
  concepts).

  The approach of overriding \<addComputedTypeAnnotations> is a last
  resort, if your logic cannot be implemented using a TreeAnnotator or a
  TypeAnnotator.  The implementation of \<addComputedTypeAnnotations> in
  \<GenericAnnotatedTypeFactory> calls the tree annotator and the type
  annotator (in that order), but by overriding the method you can cause
  your logic to be run even earlier or even later.

\end{enumerate}


\sectionAndLabel{Dataflow: enhancing flow-sensitive type refinement}{creating-dataflow}

By default, every checker performs flow-sensitive type refinement, as described
in Section~\ref{type-refinement}.

This section of the manual explains how to enhance the Checker Framework's
built-in type refinement.
Most commonly, you will inform the Checker Framework about a run-time test
that gives information about the type qualifiers in your type system.
Section~\refwithpageparen{type-refinement-runtime-tests} gives examples of
type systems with and without run-time tests.

The steps to customizing type refinement are:
\begin{enumerate}
\item{\S\ref{creating-dataflow-determine-expressions}}
  Determine which expressions will be refined.
\item{\S\ref{creating-dataflow-create-classes}}
  Create required class and configure its use.
\item{\S\ref{creating-dataflow-override-methods}}
  Override methods that handle \refclass{dataflow/cfg/node}{Node}s of interest.
\item{\S\ref{creating-dataflow-implement-refinement}}
  Implement the refinement.
\end{enumerate}

The Regex Checker's dataflow customization for the
\refmethod{checker/regex/util}{RegexUtil}{asRegex}{-java.lang.String-}
run-time check is used as a running example.

If needed, you can find more details about the implementation of
type refinement, and the control flow graph (CFG) data
structure that it uses, in the
\href{https://checkerframework.org/manual/checker-framework-dataflow-manual.pdf}{Dataflow
  Manual}.


\subsectionAndLabel{Determine expressions to refine the types of}{creating-dataflow-determine-expressions}

A run-time check or run-time
operation involves multiple expressions (arguments, results).
Determine which expression the customization will refine.  This is
usually specific to the type system and run-time test.
There is no code to write in this step; you are merely determining
the design of your type refinement.

For the program operation \code{op(a,b)}, you can refine
the types in either or both of the following ways:
\begin{enumerate}
\item Change the result type of the entire expression \code{op(a,b)}.

As an example (and as the running example of implementing a dataflow
refinement), the \code{RegexUtil.asRegex} method is declared as:

%BEGIN LATEX
\begin{smaller}
%END LATEX
\begin{Verbatim}
  @Regex(0) String asRegex(String s, int groups) { ... }
\end{Verbatim}
%BEGIN LATEX
\end{smaller}
%END LATEX

\noindent
This annotation is sound and conservative:  it says that an expression such
as \code{RegexUtil.asRegex(myString, myInt)} has type \code{@Regex(0)
  String}.  However, this annotation is imprecise.  When the \code{group}
argument is known at compile time, a better estimate can be given.  For
example, \code{RegexUtil.asRegex(myString, 2)} has type \code{@Regex(2)
  String}.

\item Change the type of some other expression, such as \code{a} or \code{b}.

As an example, consider an equality test in the Nullness type system:

\begin{Verbatim}
  @Nullable String s;
    if (s != null) {
      ...
    } else {
      ...
    }
\end{Verbatim}

The type of \<s != null> is always \<boolean>.  However, in the
true branch, the type of \<s> can be refined to \<@NonNull String>.

\end{enumerate}

If you are refining the types of arguments or the result of a method call,
then you may be able to implement your flow-sensitive refinement rules by
just writing \refqualclass{framework/qual}{EnsuresQualifier} and/or
\refqualclass{framework/qual}{EnsuresQualifierIf} annotations.
When this is possible, it is the best approach.

Sections~\ref{creating-dataflow-create-classes}--\ref{creating-dataflow-implement-refinement}
explain how to create a transfer class when the
\refqualclass{framework/qual}{EnsuresQualifier} and
\refqualclass{framework/qual}{EnsuresQualifierIf} annotations are insufficient.


\subsectionAndLabel{Create required class}{creating-dataflow-create-classes}

In the same directory as \<\emph{MyChecker}Checker.java>, create a class
named \<\emph{MyChecker}Transfer> that extends
\refclass{framework/flow}{CFTransfer}.

Leave the class body empty for now.  Your class will add functionality by
overriding methods of \<CFTransfer>, which performs the default Checker
Framework type refinement.

As an example, the Regex Checker's extended
\refclass{framework/flow}{CFTransfer} is
\refclass{checker/regex}{RegexTransfer}.

(If you disregard the instructions above and choose a different name or a
different directory for your \<\emph{MyChecker}Transfer> class, you will
also need to override the \<createFlowTransferFunction> method in your type
factory to return a new instance of the class.)

(As a reminder, use of \refqualclass{framework/qual}{EnsuresQualifier} and
\refqualclass{framework/qual}{EnsuresQualifierIf} may obviate the need for
a transfer class.)

%% More extended directions about what do to if the name is non-standard.
% If the checker's extended \refclass{framework/flow}{CFTransfer}
% starts with the name of the type system, then the type factory will use the
% transfer class without further configuration. For example, if the checker
% class is \<FooChecker>, then if the transfer class is \<FooTransfer>, then it is
% not necessary to configure the type factory
% to use \<FooTransfer>.  If some other naming convention is used, then
% to configure your checker's type factory to use the new extended
% \refclass{framework/flow}{CFTransfer}, override the
% \code{createFlowTransferFunction} method in your type factory to return a new instance
% of the extended \refclass{framework/flow}{CFTransfer}.
%
% %BEGIN LATEX
% \begin{smaller}
% %END LATEX
% \begin{Verbatim}
%   @Override
%   public CFTransfer createFlowTransferFunction(
%           CFAbstractAnalysis<CFValue, CFStore, CFTransfer> analysis) {
%       return new RegexTransfer((CFAnalysis) analysis);
%   }
% \end{Verbatim}
% %BEGIN LATEX
% \end{smaller}
% %END LATEX

%% The text below is true, but not required.
%\item \textbf{Create a class that extends
%    \refclass{framework/flow}{CFAbstractAnalysis} and uses the extended
%    \refclass{framework/flow}{CFAbstractTransfer}}
%
%  \begin{sloppypar}
%  \refclass{framework/flow}{CFAbstractTransfer} and its superclass,
%  \refclass{dataflow/analysis}{Analysis}, are the central coordinating classes
%  in the Checker Framework's dataflow algorithm. The
%  \code{createTransferFunction} method must be overridden in an extended
%  \refclass{framework/flow}{CFAbstractTransfer} to return a new instance of the
%  extended \refclass{framework/flow}{CFAbstractTransfer}.
%  \end{sloppypar}
%
%  \begin{sloppypar}
%  The Regex Checker's extended \refclass{framework/flow}{CFAbstractAnalysis} is
%  \refclass{checker/regex/classic}{RegexAnalysis}, which overrides the
%  \code{createTransferFunction} to return a new
%  \refclass{checker/regex/classic}{RegexTransfer} instance:
%  \end{sloppypar}
%
%%BEGIN LATEX
%\begin{smaller}
%%END LATEX
%\begin{Verbatim}
%  @Override
%  public RegexTransfer createTransferFunction() {
%      return new RegexTransfer(this);
%  }
%\end{Verbatim}
%%BEGIN LATEX
%\end{smaller}
%%END LATEX
%
%\item \textbf{Configure the checker's type factory to use the extended
%    \refclass{framework/flow}{CFAbstractAnalysis}}
%
%\begin{sloppypar}
%To configure your checker's type factory to use the new extended
%\refclass{framework/flow}{CFAbstractAnalysis}, override the
%\code{createFlowAnalysis} method in your type factory to return a new instance
%of the extended \refclass{framework/flow}{CFAbstractAnalysis}.
%\end{sloppypar}
%
%%BEGIN LATEX
%\begin{smaller}
%%END LATEX
%\begin{Verbatim}
%  @Override
%  protected RegexAnalysis createFlowAnalysis(
%          List<Pair<VariableElement, CFValue>> fieldValues) {
%
%      return new RegexAnalysis(checker, this, fieldValues);
%  }
%\end{Verbatim}
%%BEGIN LATEX
%\end{smaller}
%%END LATEX


\subsectionAndLabel{Override methods that handle Nodes of interest}{creating-dataflow-override-methods}

Decide what source code syntax is relevant to the run-time checks or
run-time operations you are trying to support.  The CFG (control flow
graph) represents source code as \refclass{dataflow/cfg/node}{Node}, a
node in the abstract syntax tree of the program being checked (see
\href{#creating-dataflow-representation}{``Program representation''} below).

In your extended \refclass{framework/flow}{CFTransfer}
override the visitor method that handles the \refclass{dataflow/cfg/node}{Node}s
relevant to your run-time check or run-time operation.
Leave the body of the overriding method empty for now.

For example, the Regex Checker refines the type of a run-time test method
call.  A method call is represented by a
\refclass{dataflow/cfg/node}{MethodInvocationNode}.  Therefore,
\refclass{checker/regex}{RegexTransfer} overrides the
\code{visitMethodInvocation} method:

%BEGIN LATEX
\begin{smaller}
%END LATEX
\begin{Verbatim}
  public TransferResult<CFValue, CFStore> visitMethodInvocation(
    MethodInvocationNode n, TransferInput<CFValue, CFStore> in)  { ... }
\end{Verbatim}
%BEGIN LATEX
\end{smaller}
%END LATEX


\subsubsectionAndLabel{Program representation}{creating-dataflow-representation}

% A \refclass{dataflow/cfg/node}{Node} generally maps one-to-one with a
% \refTreeclass{tree}{Tree}. When dataflow processes a method, it translates
% \refTreeclass{tree}{Tree}s into \refclass{dataflow/cfg/node}{Node}s and then
% calls the appropriate visit method on
% \refclass{framework/flow}{CFAbstractTransfer} which then performs the dataflow
% analysis for the passed in \refclass{dataflow/cfg/node}{Node}.

The \refclass{dataflow/cfg/node}{Node} subclasses can be found in the
\code{org.checkerframework.dataflow.cfg.node} package.  Some examples are
\refclass{dataflow/cfg/node}{EqualToNode},
\refclass{dataflow/cfg/node}{LeftShiftNode},
\refclass{dataflow/cfg/node}{VariableDeclarationNode}.

A \refclass{dataflow/cfg/node}{Node}
is basically equivalent to a javac compiler \refTreeclass{tree}{Tree}.

See Section~\ref{creating-javac-tips} for more information about \refTreeclass{tree}{Tree}s.
As an example, the statement \<String a = "";> is represented as this
abstract syntax tree:
\begin{Verbatim}
VariableTree:
  name: "a"
  type:
    IdentifierTree
      name: String
  initializer:
    LiteralTree
      value: ""
\end{Verbatim}



\subsectionAndLabel{Implement the refinement}{creating-dataflow-implement-refinement}

\begin{sloppypar}
Each visitor method in \refclass{framework/flow}{CFAbstractTransfer}
returns a \refclass{dataflow/analysis}{TransferResult}.  A
\refclass{dataflow/analysis}{TransferResult} represents the
refined information that is known after an operation.  It has two
components:  the result type for the \refclass{dataflow/cfg/node}{Node}
being evaluated, and a map from expressions in scope to estimates of their
types (a \refclass{dataflow/analysis}{Store}).  Each of these components is
relevant to one of the two cases in
Section~\ref{creating-dataflow-determine-expressions}:
\end{sloppypar}

\begin{enumerate}
\item
\begin{sloppypar}
Changing the \refclass{dataflow/analysis}{TransferResult}'s result type changes
the type that is returned by the \refclass{framework/type}{AnnotatedTypeFactory}
for the tree corresponding to the \refclass{dataflow/cfg/node}{Node} that was
visited.  (Remember that \refclass{common/basetype}{BaseTypeVisitor} uses the
\refclass{framework/type}{AnnotatedTypeFactory} to look up the type of a
\refTreeclass{tree}{Tree}, and then performs checks on types of one or more
\refTreeclass{tree}{Tree}s.)
\end{sloppypar}

For example, When \refclass{checker/regex}{RegexTransfer} evaluates a
\code{RegexUtils.asRegex} invocation, it updates the
\refclass{dataflow/analysis}{TransferResult}'s result type. This changes the
type of the \code{RegexUtils.asRegex} invocation when its
\refTreeclass{tree}{Tree} is looked up by the
\refclass{framework/type}{AnnotatedTypeFactory}.  See below for details.

\item
Updating the \refclass{dataflow/analysis}{Store} treats an expression as
having a refined type for the remainder of the method or conditional block. For
example, when the Nullness Checker's dataflow evaluates \code{myvar != null}, it
updates the \refclass{dataflow/analysis}{Store} to specify that the variable
\code{myvar} should be treated as having type \code{@NonNull} for the rest of the
then conditional block.  Not all kinds of expressions can be refined; currently
method return values, local variables, fields, and array values can be stored in
the \refclass{dataflow/analysis}{Store}.  Other kinds of expressions, like
binary expressions or casts, cannot be stored in the
\refclass{dataflow/analysis}{Store}.

\end{enumerate}


\begin{sloppypar}
The rest of this section details implementing the visitor method
\code{RegexTransfer.visitMethodInvocation} for the \code{RegexUtil.asRegex}
run-time test.  You can find other examples of visitor methods in
\refclass{checker/lock}{LockTransfer} and
\refclass{checker/formatter}{FormatterTransfer}.
\end{sloppypar}



\begin{enumerate}
\item \textbf{Determine if the visited \refclass{dataflow/cfg/node}{Node} is of
    interest}

A visitor method is invoked for all
instances of a given \refclass{dataflow/cfg/node}{Node} kind in the
program.
The visitor must inspect the
\refclass{dataflow/cfg/node}{Node} to determine if it is an
instance of the desired run-time test or operation.  For example,
\code{visitMethodInvocation} is called when dataflow processes any method
invocation, but the \refclass{checker/regex}{RegexTransfer} should only refine
the result of \code{RegexUtils.asRegex} invocations:

%BEGIN LATEX
\begin{smaller}
%END LATEX
\begin{Verbatim}
  @Override
  public TransferResult<CFValue, CFStore> visitMethodInvocation(...)
    ...
    MethodAccessNode target = n.getTarget();
    ExecutableElement method = target.getMethod();
    Node receiver = target.getReceiver();
    if (receiver instanceof ClassNameNode) {
      String receiverName = ((ClassNameNode) receiver).getElement().toString();

      // Is this a call to static method isRegex(s, groups) in a class named RegexUtil?
      if (receiverName.equals("RegexUtil")
          && ElementUtils.matchesElement(method,
                 "isRegex", String.class, int.class)) {
            ...
\end{Verbatim}
%BEGIN LATEX
\end{smaller}
%END LATEX

\item \textbf{Determine the refined type}

Sometimes the refined type is dependent on the parts of the operation,
such as arguments passed to it.

For example, the refined type of \code{RegexUtils.asRegex} is dependent on the
integer argument to the method call. The \refclass{checker/regex}{RegexTransfer}
uses this argument to build the resulting type \code{@Regex(\emph{i})}, where \code{\emph{i}}
is the value of the integer argument.  For simplicity the below code only uses
the value of the integer argument if the argument was an integer literal.  It
could be extended to use the value of the argument if it was any compile-time
constant or was inferred at compile time by another analysis, such as the
Constant Value Checker (\chapterpageref{constant-value-checker}).

%BEGIN LATEX
\begin{smaller}
%END LATEX
\begin{Verbatim}
  AnnotationMirror regexAnnotation;
  Node count = n.getArgument(1);
  if (count instanceof IntegerLiteralNode) {
    // argument is a literal integer
    IntegerLiteralNode iln = (IntegerLiteralNode) count;
    Integer groupCount = iln.getValue();
    regexAnnotation = factory.createRegexAnnotation(groupCount);
  } else {
    // argument is not a literal integer; fall back to @Regex(), which is the same as @Regex(0)
    regexAnnotation = AnnotationBuilder.fromClass(factory.getElementUtils(), Regex.class);
  }
\end{Verbatim}
%BEGIN LATEX
\end{smaller}
%END LATEX


\item \textbf{Return a \refclass{dataflow/analysis}{TransferResult} with the
    refined types}

Recall that the type of an expression is refined by modifying the
\refclass{dataflow/analysis}{TransferResult} returned by a visitor method.
Since the \refclass{checker/regex}{RegexTransfer} is updating the type of
the run-time test itself, it will update the result type and not the
\refclass{dataflow/analysis}{Store}.

A \refclass{framework/flow}{CFValue} is created to hold the type inferred.
\refclass{framework/flow}{CFValue} is a wrapper class for values being inferred
by dataflow:
%BEGIN LATEX
\begin{smaller}
%END LATEX
\begin{Verbatim}
  CFValue newResultValue = analysis.createSingleAnnotationValue(regexAnnotation,
      result.getResultValue().getType().getUnderlyingType());
\end{Verbatim}
%BEGIN LATEX
\end{smaller}
%END LATEX

Then, RegexTransfer's \code{visitMethodInvocation} creates and returns a
\refclass{dataflow/analysis}{TransferResult} using \code{newResultValue} as the
result type.

%BEGIN LATEX
\begin{smaller}
%END LATEX
\begin{Verbatim}
  return new RegularTransferResult<>(newResultValue, result.getRegularStore());
\end{Verbatim}
%BEGIN LATEX
\end{smaller}
%END LATEX

As a result of this code, when the Regex Checker encounters a
\code{RegexUtils.asRegex} method call, the checker will refine the return
type of the method if it can determine the value of the integer parameter
at compile time.

\end{enumerate}


\subsectionAndLabel{Disabling flow-sensitive inference}{creating-dataflow-disable}

In the uncommon case that you wish to disable the Checker Framework's
built-in flow inference in your checker (this is different than choosing
not to extend it as described in Section~\ref{creating-dataflow}), put the
following two lines at the beginning of the constructor for your subtype of
\refclass{common/basetype}{BaseAnnotatedTypeFactory}:

\begin{Verbatim}
        // disable flow inference
        super(checker, /*useFlow=*/ false);
\end{Verbatim}


\sectionAndLabel{Annotated JDK and other annotated libraries}{creating-a-checker-annotated-jdk}

You will need to supply annotations for relevant parts of the JDK;
otherwise, your type-checker may produce spurious warnings for code that
uses the JDK\@.  You have two options:

\begin{itemize}
\item
  Write JDK annotations in a fork of
  \url{https://github.com/typetools/jdk}.

  If your checker is written in a fork of
  \url{https://github.com/typetools/jdk},
  then use the same fork name (GitHub organization) and branch name;
  this is necessary so that the CI jobs use the right annotated JDK.

  Clone the JDK and the Checker Framework in the same place; that is, your
  working copy of the JDK (a \<jdk> directory) should be a sibling of your
  working copy of the Checker Framework (a \<checker-framework> directory).

  Here are some tips:
  \begin{itemize}
  \item
    Add
    an \refqualclass{framework/qual}{AnnotatedFor} annotation to each file you annotate.
  \item
    Whenever you add a file, fully annotate it, as described in
    Section~\ref{library-tips}.
  \item
    If you are only annotating fields and method signatures (but not
    ensuring that method bodies type-check), then you don't need to suppress
    warnings, because the JDK is not type-checked.
  \end{itemize}

\item
  Write JDK annotations as stub files (partial Java source files).

  Create a file \<jdk.astub> in
the checker's main source directory.  You can also create \<jdk\emph{N}.astub> files that contain methods
or classes that only exist in certain JDK versions.
The JDK stub files will be automatically used by the
checker, unless the user supplies the command-line option \<-Aignorejdkastub>.

You can also supply \<.astub> files in that directory for other libraries.
You should list those other libraries in a
\refqualclass{framework/qual}{StubFiles} annotation on the checker's main
class, so that they will also be automatically used.

When a stub file should be used by multiple checkers (for example, if it
contains purity annotations that are needed by multiple distinct checkers),
the stub file should appear in directory \<checker/src/main/resources/>.
In the distribution, it will appear at the top level of the \<checker.jar> file.
It will not be used automatically; a user must pass the
\<-Astubs=checker.jar/\emph{stubfilename.astub}> command-line argument
(see Section~\ref{annotated-libraries-using})

While creating a stub file, you may find the debugging options described in
Section~\ref{stub-troubleshooting} useful.

\end{itemize}


\sectionAndLabel{Testing framework}{creating-testing-framework}

The Checker Framework provides a convenient way to write tests for your
checker.  Each test case is a Java file, with inline indications of what
errors and warnings (if any) a checker should emit.  An example is

\begin{Verbatim}
class MyNullnessTest {
  void method() {
    Object nullable = null;
    // :: error: (dereference.of.nullable)
    nullable.toString();
  }
}
\end{Verbatim}

\noindent
When the Nullness Checker is run on the above code, it should produce
exactly one error, whose message key is \<dereference.of.nullable>, on
the line following the ``// ::'' comment.

% Don't repeat the information here, to prevent them from getting out of sync.
The testing infrastructure is extensively documented in file \ahref{https://github.com/typetools/checker-framework/blob/master/checker/tests/README}{\<checker-framework/checker/tests/README>}.

If your checker's source code is within a fork of the Checker Framework
repository, then you can copy the testing infrastructure used by some
existing type system.


\sectionAndLabel{Debugging options}{creating-debugging-options}

The Checker Framework provides debugging options that can be helpful when
implementing a checker. These are provided via the standard \code{javac} ``\code{-A}''
switch, which is used to pass options to an annotation processor.


\subsectionAndLabel{Amount of detail in messages}{creating-debugging-options-detail}

\begin{itemize}
\item \code{-AprintAllQualifiers}: print all type qualifiers, including
qualifiers meta-annotated with \code{@InvisibleQualifier}, which are
usually not shown.

\item \code{-AprintVerboseGenerics}: print more information about type
  parameters and wildcards when they appear in warning messages.  Supplying
  this also implies \code{-AprintAllQualifiers}.

\item \code{-Anomsgtext}: use message keys (such as ``\code{type.invalid}'')
rather than full message text when reporting errors or warnings.  This is
used by the Checker Framework's own tests, so they do not need to be
changed if the English message is updated.

\item \code{-AnoPrintErrorStack}: don't print a stack trace when an
internal Checker Framework error occurs.  Setting this option is rare.  You
should only do it if you have discovered a bug in a checker, you have
already reported the bug, and you want to continue using the checker on a
large codebase without being inundated in stack traces.

\item \code{-AdumpOnErrors}: Outputs a stack trace when reporting errors or warnings.

\end{itemize}


\subsectionAndLabel{Format of output}{creating-debugging-options-format}

\begin{itemize}

\item \code{-Adetailedmsgtext}: Output error/warning messages in a
  stylized format that is easy for tools to parse.  This is useful for
  tools that run the Checker Framework and parse its output, such as IDE
  plugins.  See the source code of \<SourceChecker.java> for details about
  the format.

\end{itemize}

The
\ahref{https://github.com/eisopux/javac-diagnostics-wrapper}{javac-diagnostic-wrapper}
tool can transform javac's textual output into other formats, such as JSON
in LSP (Language Server Protocol) format.


\subsectionAndLabel{Stub and JDK libraries}{creating-debugging-options-libraries}

\begin{itemize}

\item \code{-Aignorejdkastub}:
  ignore the \<jdk.astub> and \<jdk\emph{N}.astub> files in the checker directory. Files passed
  through the \code{-Astubs} option are still processed. This is useful
  when experimenting with an alternative stub file.

\item \code{-ApermitMissingJdk}:
  don't issue an error if no annotated JDK can be found.

\item \code{-AparseAllJdk}:
  parse all JDK files at startup rather than as needed.

\item \code{-AstubDebug}:
  Print debugging messages while processing stub files.
  Section~\ref{stub-troubleshooting} describes more diagnostic command-line
  arguments.

\end{itemize}

\subsectionAndLabel{Progress tracing}{creating-debugging-options-progress}

\begin{itemize}

\item \code{-Afilenames}: print the name of each file before type-checking it.
This can be useful for determining that a long compilation job is making
progress.

This option can also help to keep a Travis CI job alive (since Travis CI
terminates any job that does not produce output for 10 minutes).
This does not work if you are using Maven, because with forked compilation,
the maven-compiler-plugin queues up all the output and then prints it at the end.
(Also, the maven-compiler-plugin is buggy and sometimes doesn't print any
output if it cannot parse it.)
% Example of not producing output:
% https://github.com/codehaus-plexus/plexus-compiler/issues/66

\item \code{-Ashowchecks}: print debugging information for each
pseudo-assignment check (\<commonAssignmentCheck>) performed by
\refclass{common/basetype}{BaseTypeVisitor}; see
Section~\ref{creating-extending-visitor}.

\item \code{-AshowInferenceSteps}: print debugging information
about intermediate steps in method type argument inference
(as performed by \refclass{framework/util/typeinference}{DefaultTypeArgumentInference}).

\item \code{-AshowWpiFailedInferences}: print debugging information
about failed inference steps during whole-program inference. Must
be used with \code{-Ainfer}; see Section~\ref{whole-program-inference}.

\end{itemize}

\subsectionAndLabel{Saving the command-line arguments to a file}{creating-debugging-options-output-args}

\begin{itemize}

\item \code{-AoutputArgsToFile}:
  This saves the final command-line parameters as passed to the compiler in a file.
  The file can be used as a script to re-execute the same compilation command.
  (The script file must be marked as executable on Unix, or
  must have a \code{.bat} extension on Windows.)
  Example usage: \code{-AoutputArgsToFile=\$HOME/scriptfile}

  The \code{-AoutputArgsToFile} command-line argument is processed by
  CheckerMain, not by the annotation processor.  That means that it can be
  supplied only when you use the Checker Framework compiler (the ``Checker
  Framework javac wrapper''), and it cannot be written in a file containing
  command-line arguments passed to the compiler using the @argfile syntax.

\end{itemize}

\subsectionAndLabel{Visualizing the dataflow graph}{creating-debugging-dataflow-graph}

To understand control flow in your program and the resulting type
refinement, you can create a graphical representation of the CFG.

Typical use is:

\begin{Verbatim}
javac -processor mypackage.MyProcessor -Aflowdotdir=. MyClass.java
for dotfile in *.dot; do dot -Tpdf -o "$dotfile.pdf" "$dotfile"; done
\end{Verbatim}

\noindent
where the first command creates file \<MyClass.dot> that
represents the CFG, and the last command draws the CFG in a PDF file.
The \<dot> program is part of \ahref{http://www.graphviz.org}{Graphviz}.

In the output, conditional basic blocks are represented as octagons with
two successors.  Special basic blocks are represented as ovals (e.g., the
entry and exit point of the method).


\subsubsectionAndLabel{Creating a CFG while running a type-checker}{creating-debugging-dataflow-graph-with-typechecker}

To create a CFG while running a type-checker, use the following
command-line options. \\
(See above for typical usage.)

\begin{itemize}

\item \code{-Aflowdotdir=\emph{somedir}}:
  Specify directory for \<.dot> files visualizing the CFG\@.
  Shorthand for\\
  \<-Acfgviz=org.checkerframework.dataflow.cfg.DOTCFGVisualizer,outdir=\emph{somedir}>.
  % TODO: create the directory if it doesn't exist.
  The directory must already exist.

\item \code{-Averbosecfg}:
  Enable additional output in the CFG visualization.
  Equivalent to passing \<verbose> to \<cfgviz>, e.g. as in
  \<-Acfgviz=MyVisualizer,verbose>

\item \code{-Acfgviz=\emph{VizClassName}[,\emph{opts},...]}:
  Mechanism to visualize the control flow graph (CFG) of
  all the methods and code fragments
  analyzed by the dataflow analysis (Section~\ref{creating-dataflow}).
  The graph also contains information about flow-sensitively refined
  types of various expressions at many program points.

  The argument is a comma-separated sequence of keys or key--value pairs.
  The first argument is the fully-qualified name of the
  \<org.checkerframework.dataflow.cfg.CFGVisualizer> implementation
  that should be used. The remaining keys or key--value pairs are
  passed to \<CFGVisualizer.init>.  Supported keys include
  \begin{itemize}
  \item \<verbose>
  \item \<outdir> directory into which to write files
  \item \<checkerName>
  \end{itemize}

\end{itemize}


\subsubsectionAndLabel{Creating a CFG by running CFGVisualizeLauncher}{creating-debugging-dataflow-graph-with-cfgvisualizelauncher}

You can also use \refclass{dataflow/cfg}{CFGVisualizeLauncher} to generate a DOT
or String representation of the control flow graph of a given method in a given class.
The CFG is generated and output, but no dataflow analysis is performed.

\begin{itemize}

\item With JDK 8:

\begin{smaller}
\begin{Verbatim}
java -Xbootclasspath/p:$CHECKERFRAMEWORK/checker/dist/javac.jar \
  -cp $CHECKERFRAMEWORK/checker/dist/checker.jar \
  org.checkerframework.dataflow.cfg.CFGVisualizeLauncher \
  MyClass.java --class MyClass --method test --pdf
\end{Verbatim}
\end{smaller}


\item With JDK 11:

\begin{smaller}
\begin{Verbatim}
java -cp $CHECKERFRAMEWORK/checker/dist/checker.jar \
  org.checkerframework.dataflow.cfg.CFGVisualizeLauncher \
  MyClass.java --class MyClass --method test --pdf
\end{Verbatim}
\end{smaller}

\end{itemize}

\noindent
The above command will generate the corresponding \<.dot> and \<.pdf> files for the
method \code{test} in the class \code{MyClass} in the project directory.
To generate a string representation of the graph to standard output,
remove \<--pdf> but add \<--string>. For example (with JDK 8):

\begin{smaller}
\begin{Verbatim}
java -Xbootclasspath/p:$CHECKERFRAMEWORK/checker/dist/javac.jar \
  -cp $CHECKERFRAMEWORK/checker/dist/checker.jar \
  org.checkerframework.dataflow.cfg.CFGVisualizeLauncher \
  MyClass.java --class MyClass --method test --string
\end{Verbatim}
\end{smaller}

For more details about invoking
\refclass{dataflow/cfg}{CFGVisualizeLauncher}, run it with
no arguments.


\subsectionAndLabel{Miscellaneous debugging options}{creating-debugging-options-misc}

\begin{itemize}

\item \code{-AresourceStats}:
  Whether to output resource statistics at JVM shutdown.

\item \<-AatfDoNotCache>:
  If provided, the Checker Framework will not cache results but will
  recompute them.  This makes the Checker Framework run slower.  If the
  Checker Framework behaves differently with and without this flag, then
  there is a bug in its caching code.  Please report that bug.

\item \<-AatfCacheSize>:
  The size of the Checker Framework's internal caches.
  Ignored if \<-AatfDoNotCache> is provided.
  Most users have no need to set this.

\end{itemize}


\subsectionAndLabel{Examples}{creating-debugging-options-examples}

The following example demonstrates how these options are used:

%BEGIN LATEX
\begin{smaller}
%END LATEX
\begin{Verbatim}
$ javac -processor org.checkerframework.checker.interning.InterningChecker \
    docs/examples/InternedExampleWithWarnings.java -Ashowchecks -Anomsgtext -Afilenames

[InterningChecker] InterningExampleWithWarnings.java
 success (line  18): STRING_LITERAL "foo"
     actual: DECLARED @org.checkerframework.checker.interning.qual.Interned java.lang.String
   expected: DECLARED @org.checkerframework.checker.interning.qual.Interned java.lang.String
 success (line  19): NEW_CLASS new String("bar")
     actual: DECLARED java.lang.String
   expected: DECLARED java.lang.String
docs/examples/InterningExampleWithWarnings.java:21: (not.interned)
    if (foo == bar) {
            ^
 success (line  22): STRING_LITERAL "foo == bar"
     actual: DECLARED @org.checkerframework.checker.interning.qual.Interned java.lang.String
   expected: DECLARED java.lang.String
1 error
\end{Verbatim}
%BEGIN LATEX
\end{smaller}
%END LATEX

\subsectionAndLabel{Using an external debugger}{creating-debugging-options-external}

You can use any standard debugger to observe the execution of your checker.

You can also set up remote (or local) debugging using the following command as a template:

\begin{Verbatim}
java -jar "$CHECKERFRAMEWORK/checker/dist/checker.jar" \
    -J-Xdebug -J-Xrunjdwp:transport=dt_socket,server=y,suspend=y,address=5005 \
    -processor org.checkerframework.checker.nullness.NullnessChecker \
    src/sandbox/FileToCheck.java

\end{Verbatim}


\sectionAndLabel{Documenting the checker}{creating-documenting-a-checker}

This section describes how to write a chapter for this manual that
describes a new type-checker.  This is a prerequisite to having your
type-checker distributed with the Checker Framework, which is the best way
for users to find it and for it to be kept up to date with Checker
Framework changes.  Even if you do not want your checker distributed with
the Checker Framework, these guidelines may help you write better
documentation.

When writing a chapter about a new type-checker, see the existing chapters
for inspiration.  (But recognize that the existing chapters aren't perfect:
maybe they can be improved too.)

A chapter in the Checker Framework manual should generally have the
following sections:

\begin{description}

\item[Chapter: Belly Rub Checker]
  The text before the first section in the chapter should state the
  guarantee that the checker provides and why it is important.  It should
  give an overview of the concepts.  It should state how to run the checker.

\item[Section: Belly Rub annotations]
  This section includes descriptions of the annotations with links to the
  Javadoc.  Separate type annotations from declaration annotations, and put
  any type annotations that a programmer may not write (they are only used
  internally by the implementation) last within variety of annotation.

  Draw a diagram of the type hierarchy.  A textual description of
  the hierarchy is not sufficient; the diagram really helps readers to
  understand the system.
  The diagram will appear in directory \<docs/manual/figures/>;
  see its \<README> file for tips.

  The Javadoc for the annotations deserves the same care as the manual
  chapter.  Each annotation's Javadoc comment should use the
  \<@checker\_framework.manual> Javadoc taglet to refer to the chapter that
  describes the checker, and the polymorphic qualifier's Javadoc should
  also refer to the \<qualifier-polymorphism> section.  For example, in
  \<PolyPresent.java>:

  \begin{Verbatim}
 * @checker_framework.manual #optional-checker Optional Checker
 * @checker_framework.manual #qualifier-polymorphism Qualifier polymorphism
  \end{Verbatim}

  \noindent
  For more details, see \refclass{javacutil/dist}{ManualTaglet}.

\item[Section: What the Belly Rub Checker checks]
  This section gives more details about when an error is issued, with examples.
  This section may be omitted if the checker does not contain special
  type-checking rules --- that is, if the checker only enforces the usual
  Java subtyping rules.

\item[Section: Examples]
  Code examples.
\end{description}

Sometimes you can omit some of the above sections.  Sometimes there are
additional sections, such as tips on suppressing warnings, comparisons to
other tools, and run-time support.

You will create a new \<belly-rub-checker.tex> file,
then \verb|\input| it at a logical place in \<manual.tex> (not
necessarily as the last checker-related chapter).  Also add two references
to the checker's chapter:  one at the beginning of
chapter~\ref{introduction}, and identical text in the appropriate part of
Section~\ref{type-refinement-runtime-tests}.  Add the new file to
\<docs/manual/Makefile>.  Keep the lists in
the same order as the manual chapters, to help us notice if anything is
missing.

For a chapter or (sub)*section, use \verb|\sectionAndLabel{Section title}{section-label}|.
Section labels should start with the checker
name (as in \verb|bellyrub-examples|) and not with ``\<sec:>''.
Figure labels should start with ``fig-\emph{checkername}'' and not with ``fig:''.
These conventions are for the benefit of the Hevea program that produces
the HTML version of the manual.
Use \verb|\begin{figure}| for all figures, including those whose
content is a table, in order to have a single consistent numbering for all
figures.

Don't forget to write Javadoc for any annotations that the checker uses.
That is part of the documentation and is the first thing that many users
may see.  The documentation for any annotation should include an example
use of the annotation.
Also ensure that the Javadoc links back to the manual, using the
\<@checker\_framework.manual> custom Javadoc tag.


\sectionAndLabel{javac implementation survival guide}{creating-javac-tips}

Since this section of the manual was written, the useful ``The Hitchhiker's
Guide to javac'' has become available at
\url{http://openjdk.java.net/groups/compiler/doc/hhgtjavac/index.html}.
See it first, and then refer to this section.  (This section of the manual
should be revised, or parts eliminated, in light of that document.)


A checker built using the Checker Framework makes use of a few interfaces
from the underlying compiler (Oracle's OpenJDK javac).
This section describes those interfaces.




\subsectionAndLabel{Checker access to compiler information}{creating-compiler-information}

The compiler uses and exposes three hierarchies to model the Java
source code and classfiles.


\subsubsectionAndLabel{Types --- Java Language Model API}{creating-javac-types}

A \refModelclass{type}{TypeMirror} represents a Java type.
% Java declaration, statement, or expression.

\begin{sloppypar}
There is a \code{TypeMirror} interface to represent each type kind,
e.g., \code{PrimitiveType} for primitive types, \code{ExecutableType}
for method types, and \code{NullType} for the type of the \code{null} literal.
\end{sloppypar}

\code{TypeMirror} does not represent annotated types though.  A checker
should use the Checker Framework types API,
\refclass{framework/type}{AnnotatedTypeMirror}, instead.  \code{AnnotatedTypeMirror}
parallels the \code{TypeMirror} API, but also presents the type annotations
associated with the type.

The Checker Framework and the checkers use the types API extensively.


\subsubsectionAndLabel{Elements --- Java Language Model API}{creating-javac-elements}

An \refModelclass{element}{Element} represents a potentially-public
declaration that can be accessed from elsewhere:  classes, interfaces, methods, constructors, and
fields.  \<Element> represents elements found in both source
code and bytecode.

There is an \code{Element} interface to represent each construct, e.g.,
\code{TypeElement} for classes/interfaces, \code{ExecutableElement} for
methods/constructors, and \code{VariableElement} for local variables and
method parameters.

If you need to operate on the declaration level, always use elements rather
than trees
% in same subsection, which is the limit of the numbering.
% (Section~\ref{javac-trees})
(see below).  This allows the code to work on
both source and bytecode elements.

Example: retrieve declaration annotations, check variable
modifiers (e.g., \code{strictfp}, \code{synchronized})


\subsubsectionAndLabel{Trees --- Compiler Tree API}{creating-javac-trees}

A \refTreeclass{tree}{Tree} represents a syntactic unit in the source code,
such as a method declaration, statement, block, \<for> loop, etc. Trees only
represent source code to be compiled (or found in \code{-sourcepath});
no tree is available for classes read from bytecode.

There is a Tree interface for each Java source structure, e.g.,
\code{ClassTree} for class declaration, \code{MethodInvocationTree}
for a method invocation, and \code{ForEachTree} for an enhanced-for-loop
statement.

You should limit your use of trees. A checker uses Trees mainly to
traverse the source code and retrieve the types/elements corresponding to
them.  Then, the checker performs any needed checks on the types/elements instead.


\subsubsectionAndLabel{Using the APIs}{creating-using-the-apis}

The three APIs use some common idioms and conventions; knowing them will
help you to create your checker.

\emph{Type-checking}:
Do not use \code{instanceof Subinterface} to determine which kind of \<TypeMirror> or \<Element> an expression is,
because some of the classes that implement the TypeMirror and Element subinterfaces implement multiple
subinterfaces. For example, \<type instanceof DeclaredType> and \<type instanceof UnionType>
both return true if \<type> is an \<com.sun.tools.javac.code.Type.UnionClassType> object.

Instead, use the
\sunjavadoc{java.compiler/javax/lang/model/type/TypeMirror.html\#getKind()}{TypeMirror.getKind()}
or
\sunjavadoc{java.compiler/javax/lang/model/element/Element.html\#getKind()}{Element.getKind()}
method.  For example, if \<type> is an \<com.sun.tools.javac.code.Type.UnionClassType> object, then
\<expr.getKind() == TypeKind.DECLARED> is false and \<expr.getKind() == TypeKind.UNION> is true.


For \<Tree>s, you can use either \<Tree.getKind()> or \<instanceof>.

\emph{Visitors and Scanners}:
The compiler and the Checker Framework use the visitor pattern
extensively. For example, visitors are used to traverse the source tree
(\refclass{common/basetype}{BaseTypeVisitor} extends
\refTreeclass{util}{TreePathScanner}) and for type
checking (\refclass{framework/type/treeannotator}{TreeAnnotator} implements
\refTreeclass{tree}{TreeVisitor}).

\emph{Utility classes}:
Some useful methods appear in a utility class.  The OpenJDK convention is that
the utility class for a \code{Foo} hierarchy is \code{Foos} (e.g.,
\refModelclass{util}{Types}, \refModelclass{util}{Elements}, and
\refTreeclass{util}{Trees}).  The Checker Framework uses a common
\code{Utils} suffix to distinguish the class names (e.g., \refclass{javacutil}{TypesUtils},
\refclass{javacutil}{TreeUtils}, \refclass{javacutil}{ElementUtils}), with one
notable exception: \refclass{framework/util}{AnnotatedTypes}.


\subsubsectionAndLabel{Equality for annotations}{equality-for-annotations}

\<AnnotationMirror> is an interface that is implemented both by javac and
the Checker Framework. The documentation of
\refModelclass{element}{AnnotationMirror} says, ``Annotations should be
compared using the equals method. There is no guarantee that any particular
annotation will always be represented by the same object.''  The second
sentence is true, but the first sentence is wrong.  You should never
compare \<AnnotationMirror>s using \<equals()>, which (for some
implementations) is reference equality.
\refclass{javacutil}{AnnotationUtils} has various
methods that should be used instead. Also,
\refclass{framework/util}{AnnotationMirrorMap} and
\refclass{framework/util}{AnnotationMirrorSet} can be used.


\subsectionAndLabel{How a checker fits in the compiler as an annotation processor}{creating-checker-as-annotation-processor}

The Checker Framework builds on the Annotation Processing API
introduced in Java 6.  A type-checking annotation processor is one that extends
\refclass{javacutil}{AbstractTypeProcessor}; it gets run on each class
source file after the compiler confirms that the class is valid Java code.

The most important methods of \refclass{javacutil}{AbstractTypeProcessor}
are \code{typeProcess} and \code{getSupportedSourceVersion}. The former
class is where you would insert any sort of method call to walk the AST\@,
and the latter just returns a constant indicating that we are targeting
version 8 of the compiler. Implementing these two methods should be enough
for a basic plugin; see the Javadoc for the class for other methods that
you may find useful later on.

The Checker Framework uses Oracle's Tree API to access a program's AST\@.
The Tree API is specific to the Oracle OpenJDK, so the Checker Framework only
works with the OpenJDK javac, not with Eclipse's compiler ecj.
This also limits the tightness of
the integration of the Checker Framework into other IDEs such as \href{https://www.jetbrains.com/idea/}{IntelliJ IDEA}\@.
An implementation-neutral API would be preferable.
In the future, the Checker Framework
can be migrated to use the Java Model AST of JSR 198 (Extension API for
Integrated Development Environments)~\cite{JSR198}, which gives access to
the source code of a method.  But, at present no tools
implement JSR~198.  Also see Section~\ref{creating-ast-traversal}.



\subsubsectionAndLabel{Learning more about javac}{creating-learning-more-about-javac}

Sun's javac compiler interfaces can be daunting to a
newcomer, and its documentation is a bit sparse. The Checker Framework
aims to abstract a lot of these complexities.
You do not have to understand the implementation of javac to
build powerful and useful checkers.
Beyond this document,
other useful resources include the Java Infrastructure
Developer's guide at
\url{https://netbeans.apache.org/wiki/Java_DevelopersGuide.asciidoc} and the compiler
mailing list archives at
\url{http://mail.openjdk.java.net/pipermail/compiler-dev/}
(subscribe at
\url{http://mail.openjdk.java.net/mailman/listinfo/compiler-dev}).


\sectionAndLabel{Integrating a checker with the Checker Framework}{creating-integrating-a-checker}

% First version of how to integrate a new checker into the release.
% TODO: what steps are missing?

To integrate a new checker with the Checker Framework release, perform
the following:

\begin{itemize}

\item Make sure \code{check-compilermsgs} and \code{check-purity} run
without warnings or errors.

\end{itemize}


% LocalWords:  plugin javac's SourceChecker AbstractProcessor getMessages quals
% LocalWords:  getSourceVisitor SourceVisitor getFactory AnnotatedTypeFactory
% LocalWords:  SupportedAnnotationTypes SupportedSourceVersion TreePathScanner
% LocalWords:  TreeScanner visitAssignment AssignmentTree AnnotatedClassTypes
% LocalWords:  SubtypeChecker SubtypeVisitor NonNull isSubtype getClass nonnull
% LocalWords:  AnnotatedClassType isAnnotatedWith hasAnnotationAt TODO src jdk
% LocalWords:  processor NullnessChecker InterningChecker Nullness Nullable
% LocalWords:  AnnotatedTypeMirrors BaseTypeChecker BaseTypeVisitor basetype
% LocalWords:  Aqual Anqual CharSequence getAnnotatedType UseLovely
% LocalWords:  AnnotatedTypeMirror LovelyChecker Anomsgtext Ashowchecks enums
% LocalWords:  Afilenames dereferenced SuppressWarnings declaratively SubtypeOf
% LocalWords:  TypeHierarchy GraphQualifierHierarchy Foo qual UnknownSign
% LocalWords:  QualifierHierarchy QualifierRoot createQualifierHierarchy util
% LocalWords:  createTypeHierarchy ImplicitFor treeClasses TypeMirror Anno
% LocalWords:  LiteralTree ExpressionTree typeClasses addComputedTypeAnnotations nullable
% LocalWords:  createSupportedTypeQualifiers FooChecker nullness KeyFor
% LocalWords:  FooVisitor FooAnnotatedTypeFactory basicstyle InterningVisitor
% LocalWords:  InterningAnnotatedTypeFactory QualifierDefaults TypeKind getKind
% LocalWords:  setAbsoluteDefaults PolymorphicQualifier TreeVisitor subnodes
% LocalWords:  SimpleTreeVisitor TreePath instanceof subinterfaces TypeElement
% LocalWords:  ExecutableElement PackageElement DeclaredType VariableElement
% LocalWords:  TypeParameterElement ElementVisitor javax getElementUtils NoType
% LocalWords:  ProcessingEnvironment ExecutableType MethodTree ArrayType Warski
% LocalWords:  MethodInvocationTree PrimitiveType BlockTree TypeVisitor blog
% LocalWords:  AnnotatedTypeVisitor SimpleAnnotatedTypeVisitor html langtools
% LocalWords:  AnnotatedTypeScanner bootclasspath asType stringPatterns Foos
% LocalWords:  DefaultQualifierInHierarchy invocable wildcards novariant Utils
% LocalWords:  AggregateChecker getSupportedTypeCheckers Uninterned sourcepath
% LocalWords:  DefaultQualifier bytecode NullType strictfp ClassTree TypesUtils
% LocalWords:  ForEachTree ElementKind TreeAnnotator TreeUtils ElementUtils ecj
% LocalWords:  AnnotatedTypes AbstractTypeProcessor gcj hardcoding jsr api
% LocalWords:  typeProcess getSupportedSourceVersion fenum classpath astub
%%  LocalWords:  addAbsoluteDefault BaseAnnotatedTypeFactory superclasses
%%  LocalWords:  SupportedOptions AprintAllQualifiers InvisibleQualifier
%%  LocalWords:  Adetailedmsgtext AnoPrintErrorStack Aignorejdkastub Astubs
%%  LocalWords:  ApermitMissingJdk AstubDebug Aflowdotdir AresourceStats Regex
%%  LocalWords:  classfiles CHECKERFRAMEWORK RegexUtil asRegex myString
%%  LocalWords:  myInt CFAbstractTransfer RegexTransfer CFAbstractAnalysis
%%  LocalWords:  createTransferFunction RegexAnalysis createFlowAnalysis
%%  LocalWords:  EqualToNode LeftShiftNode VariableDeclarationNode myvar
%%  LocalWords:  MethodInvocationNode visitMethodInvocation TransferResult
%%  LocalWords:  RegexUtils LockTransfer FormatterTransfer CFValue argfile
%%  LocalWords:  RegexTransfer's newResultValue subcheckers taglet tex XXX
%%  LocalWords:  ParameterizedCheckerTest AoutputArgsToFile ManualTaglet
%%  LocalWords:  Hevea Hitchhiker's compilermsgs args Poly MyTypeSystem
%%  LocalWords:  I18nChecker i18n I18nSubchecker LocalizableKeyChecker ast
%%  LocalWords:  MyChecker MyHelperChecker getImmediateSubcheckerClasses
%%  LocalWords:  MyChecker's subchecker plugins ElementType myClass myflag
%%  LocalWords:  CheckerFrameworkTest GenericAnnotatedTypeFactory MyClass
%%  LocalWords:  addCheckedCodeDefaults RelevantJavaTypes TargetLocations
%%  LocalWords:  TypeUseLocation createExpressionAnnoHelper fromNode
%%  LocalWords:  ExpressionAnnotationHelper JavaExpressions CFTransfer LSP
%%  LocalWords:  AnnotationProvider FooTransfer createFlowTransferFunction
%%  LocalWords:  SupportedLintOptions myoption StubFiles scriptfile outdir
%%  LocalWords:  somedir Acfgviz Averbosecfg cfgviz MyVisualizer init apis
%%  LocalWords:  VizClassName CFGVisualizer MyProp MyPropChecker mypackage
%  LocalWords:  SourceFile NonNegative JavaExpression DependentTypesHelper
%  LocalWords:  createDependentTypesHelper boolean regex subclasses README
%  LocalWords:  formatter nChecker nSubchecker AprintVerboseGenerics pdf
%  LocalWords:  AshowInferenceSteps DefaultTypeArgumentInference Graphviz
%  LocalWords:  javacutil LiteralKind EnsuresQualifier EnsuresQualifierIf
%%  LocalWords:  mychecker AnnotationBuilder AnnotationUtils typequals
%%  LocalWords:  TypeAnnotationUtils reimplementing typesystem TreeType
%%  LocalWords:  getTypeFactoryOfSubchecker checkername
%%  LocalWords:  AnnotationMirror AnnotationMirrorMap AnnotationMirrorSet
%%  LocalWords:  processorpath CheckerName EnsuresNonNullIf reportError
%%  LocalWords:  reportWarning AnnotatedFor AdumpOnErrors AparseAllJdk
% LocalWords:  createTreeAnnotator ListTreeAnnotator TypeAnnotator
% LocalWords:  createTypeAnnotator ListTypeAnnotator CFGVisualizeLauncher
% LocalWords:  PropagationTreeAnnotator checkerName cfgvisualizelauncher
