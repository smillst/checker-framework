\htmlhr
\chapter{Subtyping Checker\label{subtyping-checker}}

The Subtyping Checker enforces only subtyping rules.  It operates over
annotations specified by a user on the command line.  Thus, users can
create a simple type-checker without writing any code beyond definitions of
the type qualifier annotations.

The Subtyping Checker can accommodate all of the type system enhancements that
can be declaratively specified (see Chapter~\ref{writing-a-checker}).
This includes type introduction rules (implicit
annotations, e.g., literals are implicitly considered \refqualclass{checker/nullness/qual}{NonNull}) via
the \refqualclass{framework/qual}{ImplicitFor} meta-annotation, and other features such as
flow-sensitive type qualifier inference (Section~\ref{type-refinement}) and
qualifier polymorphism (Section~\ref{qualifier-polymorphism}).

The Subtyping Checker is also useful to type system designers who wish to
experiment with a checker before writing code; the Subtyping Checker
demonstrates the functionality that a checker inherits from the Checker
Framework.

If you need typestate analysis, then you can extend a typestate checker,
much as you would extend the Subtyping Checker if you do not need typestate
analysis.  For more details (including a definition of ``typestate''), see
Chapter~\ref{typestate-checker}.
See Section~\ref{faq-typestate} for a simpler alternative.

For type systems that require special checks (e.g., warning about
dereferences of possibly-null values), you will need to write code and
extend the framework as discussed in Chapter~\ref{writing-a-checker}.


\section{Using the Subtyping Checker\label{subtyping-using}}

\begin{sloppypar}
The Subtyping Checker is used in the same way as other checkers (using the
\code{-processor org.checkerframework.common.subtyping.SubtypingChecker} option; see Chapter~\ref{using-a-checker}), except that it
requires an additional annotation processor argument via the standard
``\code{-A}'' switch. One of the two following arguments must be used with the
Subtyping Checker:
\end{sloppypar}

\begin{itemize}

\item
Provide the fully-qualified class name(s) of the annotation(s) in the custom
type system through the \code{-Aquals} option, using a comma-no-space-separated
notation:

\begin{alltt}
  javac -Xbootclasspath/p:\textit{/full/path/to/myProject/bin}:\textit{/full/path/to/myLibrary/bin} \ttbs
        -processor org.checkerframework.common.subtyping.SubtypingChecker \ttbs
        -Aquals=\textit{myModule.qual.MyQual},\textit{myModule.qual.OtherQual} MyFile.java ...
\end{alltt}

The annotations listed in \code{-Aquals} must be accessible to
the compiler during compilation in the classpath.  In other words, they must
already be compiled (and, typically, be on the javac bootclasspath)
before you run the Subtyping Checker with \code{javac}.  It
is not sufficient to supply their source files on the command line.

\item
Provide the fully-qualified paths to a set of directories that contain the
annotations in the custom type system through the \code{-AqualDirs} option,
using a colon-no-space-separated notation. For example:

\begin{alltt}
  javac -Xbootclasspath/p:\textit{/full/path/to/myProject/bin}:\textit{/full/path/to/myLibrary/bin} \ttbs
        -processor org.checkerframework.common.subtyping.SubtypingChecker \ttbs
        -AqualDirs=\textit{/full/path/to/myProject/bin}:\textit{/full/path/to/myLibrary/bin} MyFile.java
\end{alltt}

Note that in these two examples, the compiled class file of the
\<myModule.qual.MyQual> and \<myModule.qual.OtherQual> annotations must exist
in either the \<myProject/bin> directory or the \<myLibrary/bin> directory. The
following placement of the class files will work with the above commands:

\begin{alltt}
  .../myProject/bin/myModule/qual/MyQual.class
  .../myLibrary/bin/myModule/qual/OtherQual.class
\end{alltt}

The two options can be used at the same time to provide groups of annotations
from directories, and individually named annotations.

\end{itemize}

To suppress a warning issued by the Subtyping Checker, use a
\sunjavadocanno{java/lang/SuppressWarnings.html}{SuppressWarnings}
annotation, with the argument being the unqualified, uncapitalized name of
any of the annotations passed to \code{-Aquals}.  This will suppress all
warnings, regardless of which of the annotations is involved in the
warning.  (As a matter of style, you should choose one of the annotations
as your \code{@SuppressWarnings} key and stick with it for that entire type
hierarchy.)


\section{Subtyping Checker example\label{subtyping-example}\label{encrypted-example}}

Consider a hypothetical \code{Encrypted} type qualifier, which denotes that the
representation of an object (such as a \code{String}, \code{CharSequence}, or
\code{byte[]}) is encrypted. To use the Subtyping Checker for the \code{Encrypted}
type system, follow three steps.

\begin{enumerate}
\item
Define two annotations for the \code{Encrypted} and \code{PossiblyUnencrypted} qualifiers:

% alltt because it uses \textit
\begin{alltt}
package \textit{myModule}.qual;

import org.checkerframework.framework.qual.*;
import java.lang.annotation.ElementType;
import java.lang.annotation.Target;

/**
 * Denotes that the representation of an object is encrypted.
 */
@SubtypeOf(PossiblyUnencrypted.class)
@ImplicitFor(literal= { LiteralKind.NULL })
@DefaultFor({TypeUseLocation.LOWER_BOUND})
@Target({ElementType.TYPE_USE, ElementType.TYPE_PARAMETER})
public @interface Encrypted {}
\end{alltt}

% alltt because it uses \textit
\begin{alltt}
package \textit{myModule}.qual;

import org.checkerframework.framework.qual.DefaultQualifierInHierarchy;
import org.checkerframework.framework.qual.SubtypeOf;
import java.lang.annotation.ElementType;
import java.lang.annotation.Target;

/**
 * Denotes that the representation of an object might not be encrypted.
 */
@DefaultQualifierInHierarchy
@SubtypeOf({})
@Target({ElementType.TYPE_USE, ElementType.TYPE_PARAMETER})
public @interface PossiblyUnencrypted {}
\end{alltt}

Note that all custom annotations must have the
\<@Target({ElementType.TYPE\_USE})> meta-annotation. See section
\ref{define-type-qualifiers}.

Don't forget to compile these classes:

\begin{Verbatim}
$ javac myModule/qual/Encrypted.java myModule/qual/PossiblyUnencrypted.java
\end{Verbatim}

The resulting \<.class> files should either be on your classpath, or on the
processor path (set via the \<-processorpath> command-line option to javac).

\item
  Write \code{@Encrypted} annotations in your program (say, in file
  \code{YourProgram.java}):

\begin{alltt}
import \textit{myModule}.qual.Encrypted;

...

public @Encrypted String encrypt(String text) {
    // ...
}

// Only send encrypted data!
public void sendOverInternet(@Encrypted String msg) {
    // ...
}

void sendText() {
    // ...
    @Encrypted String ciphertext = encrypt(plaintext);
    sendOverInternet(ciphertext);
    // ...
}

void sendPassword() {
    String password = getUserPassword();
    sendOverInternet(password);
}
\end{alltt}

You may also need to add \code{@SuppressWarnings} annotations to the
\code{encrypt} and \code{decrypt} methods.  Analyzing them is beyond the
capability of any realistic type system.

\item
  Invoke the compiler with the Subtyping Checker, specifying the
  \code{@Encrypted} annotation using the \code{-Aquals} option.
  You should add the \code{Encrypted} classfile to the processor classpath:

\begin{alltt}
  javac -processorpath \textit{myqualpath} -processor org.checkerframework.common.subtyping.SubtypingChecker \
        -Aquals=\textit{myModule.qual.Encrypted},\textit{myModule.qual.PossiblyUnencrypted} YourProgram.java

YourProgram.java:42: incompatible types.
found   : @myModule.qual.PossiblyUnencrypted java.lang.String
required: @myModule.qual.Encrypted java.lang.String
    sendOverInternet(password);
                     ^
\end{alltt}

\item
You can also provide the fully-qualified paths to a set of directories
that contain the qualifiers using the \code{-AqualDirs} option, and add
the directories to the boot classpath, for example:

\begin{alltt}
  javac -Xbootclasspath/p:\textit{/full/path/to/myProject/bin}:\textit{/full/path/to/myLibrary/bin} \ttbs
        -processor org.checkerframework.common.subtyping.SubtypingChecker \ttbs
        -AqualDirs=\textit{/full/path/to/myProject/bin}:\textit{/full/path/to/myLibrary/bin} YourProgram.java
\end{alltt}

\begin{sloppypar}
Note that in these two examples, the compiled class file of the
\<myModule.qual.Encrypted> and \<myModule.qual.PossiblyUnencrypted> annotations
must exist in either the \<myProject/bin> directory or the \<myLibrary/bin>
directory. The following placement of the class files will work with the above
commands:
\end{sloppypar}

\begin{alltt}
  .../myProject/bin/myModule/qual/Encrypted.class
  .../myProject/bin/myModule/qual/PossiblyUnencrypted.class
\end{alltt}

\end{enumerate}

Also, see the example project in the \<checker/examples/subtyping-extension> directory.


% LocalWords:  TODO ImplicitFor Aquals sourcepath java NonNull AqualDirs
% LocalWords:  CharSequence classpath nullness quals SuppressWarnings classfile
% LocalWords:  uncapitalized processorpath Warski MyFile YourProgram
%%  LocalWords:  bootclasspath PossiblyUnencrypted myProject myLibrary
%%  LocalWords:  ElementType myqualpath sendOverInternet
